\chapter{Modelling in mechanics}
\section{Modelling assumptions}
\begin{tabular}{|m{3.9cm} | m{13.5cm}|}
	\hline
	\textbf{Model} & \textbf{Assumptions} \\
	\hline
	Particle & Mass of the object is concentrated at a single point, rotational effect of external forces and air resistance can be ignored, volume is negligible \\
	\hline
	Rod & Mass is concentrated along a line, no thickness, rigid \\
	\hline
	Lamina & Mass is distributed across a flat surface \\
	\hline
	Uniform body & Mass is concentrated at the centre of mass \\
	\hline
	Light object & Treat the object as if it has zero mass, tension is the same at both ends of the string  \\
	\hline
	Inextensible string / rod & Tension / thrust is the same at any point on the string / rod, any stretching effect can be ignored, \textbf{same acceleration and velocity throughout the system} \\
	\hline
	Smooth surface & No friction between the surface and other objects\\
	\hline
	Rough surface &  Objects experience a frictional force if they are moving or acted on by a force  \\
	\hline
	Wire &  Treat as one-dimensional, doesn't bend (rigid) \\
	\hline
	Smooth and light pulley & Pulley has no mass, tension is the same on either side of the pulley, no friction around the pulley \\
	\hline
	Bead & Mores freely along a wire or string, tension is the same on either side  \\
	\hline
	Peg & Dimensionless and fixed, can be rough or smooth \\
	\hline
	Air resistance & Usually negligible  \\
	\hline
	Gravity &  All objects with mass are attracted towards the Earth, gravity is uniform and acts vertically downwards, $g$ is constant and is taken as $9.8 \: \mathrm{m} \: \mathrm{s}^{-2}$ unless otherwise stated \\
	\hline
\end{tabular}

\chapter{Constant acceleration}
\section{SUVAT equations}
\begin{itemize}
	\item $s=ut+\dfrac{1}{2}at^2$
	\item $s=vt-\dfrac{1}{2}at^2$
	\item $v=u+at$
	\item $v^2=u^2+2as$
	\item $s=\dfrac{1}{2}(u+v)t$
\end{itemize}

\chapter{Forces and motion}
\section{Types of forces}
\begin{description}
	\item[Weight:] $W=mg$
	\item[Normal contact force:] symbol = $R$ or $N$
	\item[Static friction:] Depends on driving force, $F\leq \mu R$
	\item[Dynamic friction:] $f=\mu R$ ($\mu$=coefficient of kinetic friction), exists on \textbf{rough surfaces}
	\item[Thrust / compression:] Object being pushed along using a light rod
	\item[Tension:] $T=\text{elastic coefficient}\times\text{extension}=k\times\Delta x$
	\item[Air resistance / drag:] resistance due to air / water / fluid
	\item[Driving / propulsive force:] forward force produced by the object itself
\end{description}

\section{Common scenarios}
\subsection{Connected particles}
\begin{itemize}
	\item Acceleration is the same across the whole system
	\item Internal force can be ignored
	\item Tension at the same rope has the same magnitude
\end{itemize}

\subsection{Lift}
\begin{itemize}
	\item Consider the whole system to find tension in the string
	\item Consider one object only to find force they exerted on each other
	\item Rising: $R-W=ma$
	\item Moving down: $W-R=ma$
	\item On rest: $R=W$
\end{itemize}

\subsection{Fixed pulley}
\begin{itemize}
	\item Same tension
	\item Same magnitude for acceleration (different direction)
	\item Use simultaneous equations to find tension
	\item $\text{Force on pulley} = 2 \times \text{tension}$
\end{itemize}


\chapter{Variable acceleration}
\section{Finding distance travelled}
\begin{itemize}
	\item Use graph to show sign changes during the interval
	\item Remember to account for periods with negative velocity
\end{itemize}