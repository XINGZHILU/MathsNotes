\section{Notation}
\begin{itemize}
	\item $X \sim N(\mu,\sigma^2)$
	\item $\mu$ = mean of the population
	\item $\sigma^2$ = \textbf{variance} of the data
\end{itemize}
\section{Properties}
\begin{itemize}
	\item The data is \textbf{continuous}
	\item Has parameters $\mu$ (mean) and $\sigma^2$ (variance)
	\item Is symmetrical: mean = median = mode
	\item Has a bell-shaped curve with asymptotes at each end
	\item Total area under the curve = $1$
	\item Has points of inflection at $\mu+\sigma$ and $\mu-\sigma$
\end{itemize}

\section{Estimating probabilities}
\begin{itemize}
	\item $68\%$ of observations lie within $\pm 1$ standard deviation of the mean
	\item $95\%$ of observations lie within $\pm 2$ standard deviation of the mean
	\item $99.8\%$ of observations lie within $\pm 3$ standard deviation of the mean
\end{itemize}
\includegraphics[width=0.7\linewidth]{images/normal_estimate}

\section{Approximation of binomial distribution}
If $n$ is large ($n\geq35$) and $p$ is close to $0.5$, then $X \sim B(n,p)$ can be modelled as $$Y \sim N(np, np(1-p))$$
\subsection{Approximations}
\begin{itemize}
	\item $\text{P}(X\geq a)\approx \text{P}(Y\geq [a-0.5])$
	\item $\text{P}(X=a)\approx \text{P}([a-0.5]<Y<[a+0.5])$
	\item $\text{P}(X \leq a)\approx \text{P}(Y\leq [a+0.5])$
\end{itemize}


\section{Sample mean}
If $n$ is large enough ($n\geq35$) and $X\sim N(\mu, \sigma^2)$, then sample mean $\overline{X}$ is 
normally distributed: 
$$\overline{X} \sim N\left(\mu, \frac{\sigma^2}{n}\right)=N\left(\mu, \left(\frac{\sigma}{\sqrt{n}}\right)^2\right)$$