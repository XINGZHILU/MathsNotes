\section{Definitions}
\begin{description}
    \item[Bivariate data] Data which has pairs of related values
    \item[Independent / explanatory variable] The variable that the researcher can control, usually plotted on the x-axis
    \item[Dependent / response variable] The variable that the researcher measures, usually plotted on the y-axis
    \item[Correlation] Describes the nature of the linear relationship between 2 variables
\end{description}

\section{Causal relationships}
\begin{itemize}
    \item 2 variables have a casual relationship if a change in 1 variable causes a change in the other
    \item[$\star$] Correlation doesn't mean causation (add some explanations in context for questions)
\end{itemize}

\section{Linear regression}
* Work these out using a \textbf{calculator} in exams
\subsection{Regression equation for least squares regression line}
\begin{itemize}
    \item Regression line of $y$ on $x$: $y=a+bx$
    \item $b=\dfrac{\sum (x_i-\overline{x}) (y_i-\overline{y})}{\sum (x_i-\overline{x})^2}$ \textbf{(not needed for the exam)}
    \item $a=\overline{y}-b\overline{x}$
    \item Positive correlation: $b$ positive, negative correlation: $b$ negative
\end{itemize}

\subsection{Predicting values}
\begin{itemize}
    \item Should not extrapolate, only do interpolation
    \item Reliability: reliable as it is within the range of data / not reliable as it is extrapolating
    \item[$\star$] Not suitable for predicting $x$ based on $y$ (the independent variable in this model is $x$, you should not use this model to predict the value of $x$ based on $y$)
\end{itemize}

\subsection{Reason for using a regression line}
\begin{itemize}
    \item The data shows a strong (positive / negative) linear correlation
\end{itemize}
