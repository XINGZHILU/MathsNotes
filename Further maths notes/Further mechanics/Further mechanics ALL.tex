\documentclass[oneside, fleqn, 11pt]{book}
\usepackage[a4paper, total={7.2in, 10.5in}]{geometry}
\usepackage{tikz}
\usetikzlibrary{calc}
\usepackage{setspace}
\usepackage{graphicx}
\usepackage{amsmath}
\DeclareMathOperator\cis{cis}
\usepackage{pgfplots}
\graphicspath{ {./images/} }
\usepackage{bookmark}
\setcounter{tocdepth}{0}

%\usepackage[T1]{fontenc}
\usepackage{mathptmx}

\hypersetup{
	colorlinks   = true, %Colours links instead of ugly boxes
	urlcolor     = blue, %Colour for external hyperlinks
	linkcolor    = black, %Colour of internal links
	citecolor   = red %Colour of citations
}

\usepackage{hyperref}
\usepackage{blindtext}
\counterwithin*{chapter}{part}
\newcommand*{\Part}[2][\partheading]{%
  \refstepcounter{part}%
  \def\partheading{#2}%
  \part*{#2}%
  \addcontentsline{toc}{part}{#1}%
}

\newcommand{\tikzAngleOfLine}{\tikz@AngleOfLine}
\def\tikz@AngleOfLine(#1)(#2)#3{%
	\pgfmathanglebetweenpoints{%
		\pgfpointanchor{#1}{center}}{%
		\pgfpointanchor{#2}{center}}
	\pgfmathsetmacro{#3}{\pgfmathresult}%
}

\title{Further Mechanics Notes}
\author{Xingzhi Lu}
\date{}

\begin{document}
	\maketitle
    \everymath{\displaystyle}
    \tableofcontents
	
	\chapter{Momentum and impulse}
    \section{Finding areas of shapes}
\begin{itemize}
    \item Area of triangle $\bigtriangleup ABC$ = $\dfrac{1}{2}|\overrightarrow{AB}\times\overrightarrow{AC}|$
    \item Area of parallelogram $ABCD$ = $|(b-a) \times (d-a)|=|(a\times b)+(b \times d) + (d \times a)|$ ($A$, $B$, $C$, $D$ have position vector $a$, $b$, $c$, $d$ respectively)
\end{itemize}
\section{Scalar triple product}
\begin{itemize}
    \item Volume of parallelepiped = $a\cdot(b\times c)$ ($a$, $b$, $c$ = 3 different sides)
    \item Volume of tetrahedron $ABCD$ = $\dfrac{1}{6}|\overrightarrow{AD}\cdot(\overrightarrow{AB}\times\overrightarrow{AC})|$
\end{itemize}
\section{Straight lines}
\subsection{Vector equation of line}
\begin{itemize}
    \item $(\vec{r}-\vec{a})\times\vec{b}=0$
    \item $\vec{a}$ = position vector of a point on line, $\vec{b}$ = directional vector
\end{itemize}
\subsection{Direction cosines}
For straight line $(r-a)\times b=0$, where $a=x\vec{i}+y\vec{j}+z\vec{k}$ and the line makes angle $\alpha$, $\beta$ and $\gamma$ with the positive $x$-, $y$- and $z$-axes respectively:
\begin{itemize}
    \item $l=\cos\alpha = \dfrac{x}{|a|}$
    \item $m=\cos\beta = \dfrac{y}{|a|}$
    \item $n=\cos\gamma = \dfrac{z}{|a|}$
    \item $l^2+m^2+n^2=1$
\end{itemize}
	
    \chapter{Work, energy and power}
    % \section{Circle}
% \subsection{Definition of circle}
% \subsubsection{Cartesian form}
% \begin{itemize}
%     \item $(x-a)^2+(y-b)^2=r^2$
% \end{itemize}

% \subsubsection{Parametric form}
% \begin{itemize}
%     \item $x=a+r\cos\theta$
%     \item $y=b+r\sin\theta$
% \end{itemize}

% \subsubsection{Polar form}
% \begin{itemize}
%     \item $r=a$
%     \item $r=a\sin\theta$
%     \item $r=a\cos\theta$
% \end{itemize}

\section{Parabola}
\subsection{Graph}
\includegraphics[width=0.6\textwidth]{parabola.png}
\begin{itemize}
    \item Symmetrix about the $x$-axis
    \item Focus at $(a, 0)$
    \item Vertex at $(0, 0)$
\end{itemize}
\subsection{Definition}
\begin{itemize}
    \item The locus of points that are the \textbf{same distance} from a fixed point, $S$,
          called the \textbf{focus}, and a fixed straight line called the \textbf{directrix}
    \item $\dfrac{\text{distance to foci}}{\text{distance to directrix}} = e = 1$
\end{itemize}
\subsection{Cartesian equation}
\begin{itemize}
    \item $y^2=4ax$ ($a>0$)
\end{itemize}
\subsection{Parametric equation}
\begin{itemize}
    \item $x=at^2$
    \item $y=2at$
    \item $t\in \Rset$
\end{itemize}
\subsection{Eccentricity}
\begin{itemize}
    \item $e=1$
\end{itemize}
\subsection{Directrix}
\begin{itemize}
    \item The directrix has equation $x+a=0$
\end{itemize}
\subsection{Tangents and normals}
\begin{itemize}
    \item $\dfrac{\dy}{\dx} = \frac{1}{t} = \frac{2a}{y}$
    \item Equation of tangent: $ty=x+at^2$ at $P(at^2, 2at)$
    \item Equation of normal: $y+tx=2at+at^3$ at $P(at^2, 2at)$
\end{itemize}

\section{Rectangular hyperbolas}
\subsection{Graph}
\includegraphics[width=0.3\textwidth]{rectangular_hyperbola.png}
\begin{itemize}
    \item Asymptotes at $x=0$ and $y=0$ ($x$ and $y$-axis)
\end{itemize}
\subsection{Definition}
\begin{itemize}
    \item The locus of points that are the \textbf{same distance} from a fixed point, $S$,
          called the \textbf{focus}, and a fixed straight line called the \textbf{directrix}
    \item $\dfrac{\text{distance to foci}}{\text{distance to directrix}} = e = 1$
\end{itemize}
\subsection{Cartesian equation}
\begin{itemize}
    \item $xy=c^2$ ($c>0$)
\end{itemize}
\subsection{Parametric equation}
\begin{itemize}
    \item $x=ct$
    \item $y=\frac{c}{t}$
    \item $t \neq 0, t\in\Rset$
\end{itemize}
\subsection{Eccentricity}
\begin{itemize}
    \item $e=\sqrt{2}$
\end{itemize}
\subsection{Directrix}
\begin{itemize}
    \item $x+y=\pm c\sqrt{2}$
\end{itemize}
\subsection{Tangents and normals}
\begin{itemize}
    \item Equation of tangent: $x+t^2y=2ct$ at $P(ct, \frac{c}{t})$
    \item Equation of normal: $t^3x-ty=c(t^4-1)$ at $P(ct, \frac{c}{t})$
\end{itemize}
	
	\chapter{Elastic strings and springs and elastic energy}
	\section{Hooke's law}
\begin{itemize}
    \item $\text{Tension produced}\propto x$ $\rightarrow$ $T=kx$, where $k$ is a constant
    \item $k$ depends on the unstretched length of the string or spring ($l$) and the \textbf{modulus of elasticity of the string or spring} ($\lambda$, unit = N)
    \item Hence $T=\frac{\lambda x}{l}$
    \item[*] Can also be applied if the string or spring is compressed
\end{itemize}

\section{Elastic energy}
\begin{itemize}
    \item Work done in stretching an elastic string or spring of modulus of elasticity $\lambda$ from its natural length $l$ to a length of $(l+x)$ = $\frac{\lambda x^2}{2l}$
    \item Elastic potential energy stored = amount of energy used to stretch the spring to a length of $(l+x)$ = $\frac{1}{2}kx^2$ = $\frac{\lambda x^2}{2l}$
    \item[*] Can also be applied when an elastic string or spring is compressed
\end{itemize}
	
	
	\chapter{Elastic collisions in one dimension}
    \section{Vieta's Law}
For $a_{n}x^{n}+a_{n-1}x^{n-1}+\cdots +a_{1}x+a_{0}=0$:
\begin{itemize}
    \item $\sum x_i=-\frac{a_{n-1}}{a_n}$
    \item $\sum x_ix_j=\frac{a_{n-2}}{a_n}$
    \item $\sum x_ix_jx_k = -\frac{a_{n-3}}{a_n}$
    \item $\sum _{1\leq i_{1}<i_{2}<\cdots <i_{k}\leq n}\left(\prod _{j=1}^{k}r_{i_{j}}\right)=(-1)^{k}{\frac {a_{n-k}}{a_{n}}}$
    \item $\prod x_i=(-1)^n\frac{a_0}{a_n}$
\end{itemize}
	
	\chapter{Elastic collisions in two dimensions}
	\section{Resolving forces}
\begin{enumerate}
    \item Resolve perpendicular and parallel to the direction of motion
    \item Clearly indicate the direction of motion
    \item Use $\sum F = ma$
\end{enumerate}

\section{Friction}
\begin{itemize}
    \item $F_r \leq \mu R$
    \item Friction is only \textbf{as large as it needs to be} to oppose the motion
    \item If the object is moving then $F_r = \mu R$
\end{itemize}


	
	
	
	
	
\end{document}