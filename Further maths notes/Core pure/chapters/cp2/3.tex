

\section{Differentiation and integration of inverse trigonometric functions }
\begin{tabular}{ll}
    $f(x)$                      & $\int f(x) \: dx$                                                                                     \\
    $\dfrac{1}{\sqrt{a^2-x^2}}$ & $\arcsin\left(\dfrac{x}{a}\right)$                                                                    \\
    $\dfrac{1}{a^2+x^2}$        & $\dfrac{1}{a}\arctan\left(\dfrac{x}{a}\right)$                                                        \\
    $\dfrac{1}{\sqrt{x^2-a^2}}$ & $\text{arcosh}\left(\dfrac{x}{a}\right)$, $\ln\left(x+\sqrt{x^2-a^2}\right) (x>a)$                    \\
    $\dfrac{1}{\sqrt{x^2-a^2}}$ & $\text{arsinh}\left(\dfrac{x}{a}\right)$, $\ln\left(x+\sqrt{x^2+a^2}\right)$                          \\
    $\dfrac{1}{a^2-x^2}$        & $\dfrac{1}{2a}\ln \left|\dfrac{a+x}{a-x}\right| = \dfrac{1}{a}\text{artanh}\left(\dfrac{x}{a}\right)$ \\
    $\dfrac{1}{x^2-a^2}$        & $\dfrac{1}{2a}\ln \left|\dfrac{x-a}{x+a}\right|$                                                      \\
\end{tabular}

\section{Trig substitution for square roots}
\begin{itemize}
    \item $\sqrt{a(x+b)^2-c}$
          \begin{description}
              \item[Substitution:] $x = \sqrt{\dfrac{c}{a}}\sec\theta - b$
          \end{description}
    \item $\sqrt{a(x+b)^2+c}$
          \begin{description}
              \item[Substitution:] $x = \sqrt{\dfrac{c}{a}}\tan\theta - b$
          \end{description}
    \item $\sqrt{-a(x+b)^2-c}$
          \begin{description}
              \item[Substitution:] $x = \sqrt{\dfrac{c}{a}}\cos\theta - b$ or $\sqrt{\dfrac{c}{a}}\sin\theta - b$
          \end{description}

\end{itemize}

\section{Integration techniques}
\subsection{Integrating to other functions}
\[\int \frac{1}{\sqrt{a^2-x^2}} \dx=\arcsin \left(\frac{x}{a}\right) + c \]
\[\int \frac{1}{a^2+x^2} \dx= \frac{1}{a}\arctan \left(\frac{x}{a}\right) + c \]
\[\int \frac{1}{\sqrt{x^2-a^2}} \dx= \arcosh \left(\frac{x}{a}\right) + c = \ln \left(x + \sqrt{x^2-a^2}\right) + c \:\:  (x>a)\]
\[\int \frac{1}{\sqrt{a^2+x^2}} \dx= \arsinh \left(\frac{x}{a}\right) + c = \ln \left(x + \sqrt{x^2+a^2}\right) + c \]
\[\int \frac{1}{a^2-x^2} \dx=\frac{1}{a}\artanh \left(\frac{x}{a}\right) + c = \frac{1}{2a}\ln \left|\frac{a+x}{a-x}\right| + c\]
\[\int \frac{1}{x^2-a^2} \dx= \frac{1}{2a}\ln \left|\frac{x-a}{x+a}\right| + c\]
\subsection{Improper integrals with infinite range}
\subsection{Improper integrals with integrand undefined at a value in the range}
\subsection{Integrate using partial functions}
\section{Mean value of a function}
Mean of $f(x)$ over the interval $[a,b]$ = $\dfrac{1}{b-a}\int_{a}^{b} f(x) \: dx$

\subsection{Trigonometric integration}
\subsubsection{$\sin$ and $\cos$}


\subsubsection{$\sec$ and $\tan$}
\[\int \sec^n x \dx = \int \sec^{n-2} \sec^2 x \dx = \int \sec^{n-2} x \: \mathrm{d} \tan x\]