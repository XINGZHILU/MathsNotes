

\section{Taylor and Maclaurin series of expansion}
\subsection{Taylor series}
$$f(x)=f(0)+f'(a)x+\frac{f''(a)(x)-a^2}{2!}+\dots+\frac{f^{(r)}(a)(x-a)^r}{r!}+\dots$$

\subsection{Maclaurin series}
$$f(x)=f(0)+f'(0)x+\frac{f''(0)x^2}{2!}+\dots+\frac{f^{(r)}(0)x^r}{r!}+\dots$$
This series is valid provided that $f(0), f'(0), f''(0),\dots,f^{(r)}(0),\dots$ all have \textbf{finite values}

\section{Series expansion of compound functions}
\begin{itemize}
	\item $e^{x}=1+x+\frac{x^{2}}{2!}+...+\frac{x^{r}}{r!}+...$ for all $x$
	\item $\ln(1+x) = x - \frac{x^{2}}{2} + \frac{x^{3}}{3} + ... + (-1)^{r+1}\frac{x^{r}}{r!} +...$ for $-1<x\leq1$
	\item $\sin x = \frac{e^{ix}-e^{-ix}}{2i}=x-\frac{x^{3}}{3!}+\frac{x^{5}}{5!}+...+(-1)^r\dfrac{x^{2r+1}}{(2r+1)!}+\dots$ for all $x$
	\item $\cos x = \frac{e^{ix}+e^{-ix}}{2}=1-\frac{x^{2}}{2!}+\frac{x^{4}}{4!}-...+(-1)^r\dfrac{x^{2r}}{(2r)!}+\dots$ for all $x$
	\item $\arctan x = x-\dfrac{x^3}{x}+\dfrac{x^5}{5}-\dots+(-1)^r\dfrac{x^{2r+1}}{2r+1}+\dots$ for $-1<x\leq1$
\end{itemize}

\section{Proving series properties}
\begin{enumerate}
	\item Use Taylor and Maclaurin Series
	\item Use basic formulae for expansion
	\item Use geometric series
\end{enumerate}

\section{Testing for convergence}
\subsection{$n$th term test}
\begin{itemize}
    \item $\lim_{n\rightarrow\infty}a_n \neq 0$ $\rightarrow$ $\sum_{n=1}^{\infty}a_n$ diverges
    \item $\sum_{n=1}^{\infty}a_n$ converges $\rightarrow$ $\lim_{n\rightarrow\infty}a_n = 0$
\end{itemize}

\subsection{Integral test}
\begin{itemize}
    \item If $a_n$ decrease and $a_n>0$, $\sum_{n=1}^{\infty}a_n$ and $\int_{1}^{\infty} f(x) \dx$
    has the same properties of convergence or divergence
\end{itemize}

\subsection{Comparison test}
Suppose $b_n<a_n$ for all $n$:
\begin{itemize}
    \item $\sum_{n=1}^{\infty}b_n$ diverges $\rightarrow$ $\sum_{n=1}^{\infty}a_n$ diverges
    \item \item $\sum_{n=1}^{\infty}a_n$ converges $\rightarrow$ $\sum_{n=1}^{\infty}b_n$ converges
    \item Compare $a_n$ with $p$-series and geometrical series
    \begin{description}
        \item[$p$-series] $\sum_{n=1}^{\infty} \left(\frac{1}{n}\right)^p$ is divergent if $p\leq 1$ and 
        convergent if $p>1$
    \end{description}
\end{itemize}

\subsection{Root test}
\begin{itemize}
    \item $\lim_{n\rightarrow\infty}\sqrt[n]{a_n} < 1$ $\rightarrow$ $a_n$ converge
    \item \item $\lim_{n\rightarrow\infty}\sqrt[n]{a_n} > 1$ $\rightarrow$ $a_n$ diverge
\end{itemize}

\subsection{Ratio test}
\begin{itemize}
    \item $\lim_{n\rightarrow\infty} \left|\frac{a_{n+1}}{a_n}\right| < 1$: convergent
    \item $\lim_{n\rightarrow\infty} \left|\frac{a_{n+1}}{a_n}\right| > 1$: divergent
    \item $\lim_{n\rightarrow\infty} \left|\frac{a_{n+1}}{a_n}\right| = 1$: not sure
\end{itemize}

\section{Summation of series}
\begin{itemize}
    \item Try to break down $a_n$ into the form $a_n=b_{n+1}-b_n$
\end{itemize}