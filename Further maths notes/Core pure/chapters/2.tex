\subsection{Forms expressing complex number}
\begin{tikzpicture}
	\draw[->, thick] (-0.5,0)--(4,0) node[right]{$x$};
	\draw[->, thick] (0,-0.5)--(0,3.5) node[above]{$y$};
	\coordinate (O) at (0,0);
	\coordinate (A) at (3.5,0);
	\coordinate (B) at (3.5,3);
	\draw [dotted] (A) -- (B);
	\draw [->] (O) -- (B);
	\node[below] at (2,0) {a};
	\node [right] at (3.5,1.5) {b};
	\node [above left] at (1.75, 1.5) {r};
	\tikzAngleOfLine(O)(A){\AngleStart}
	\tikzAngleOfLine(O)(B){\AngleEnd}
	\draw[black,<->] (O)+(\AngleStart:0.7cm) arc (\AngleStart:\AngleEnd:0.7cm);
	\node [above right] at (0.65, 0) {$\theta$};
\end{tikzpicture}
\subsubsection{Cartesian form}
$z=a+bi$\\
$\operatorname{Re}(z) = a$, $\operatorname{Im}(z) = b$\\
$|z|=\sqrt{a^2+b^2}$
\subsubsection{Polar form}
$z=(r,\theta)$\\
$r=\sqrt{a^2+b^2}$, $\theta = \arg(z)$\\
$a=r\cos\theta$, $b=r\sin\theta$, $\tan\theta=\dfrac{b}{a}$\\
$z=r\cos\theta+r\sin\theta i = r(\cos\theta+\sin\theta i)=r\cis \theta$

\subsubsection{Exponential / Euler form}
$z=re^{i\theta}$
\subsubsection{When to use which form}
\begin{description}
	\item[Addition / subtraction: ] Cartesian form
	\item[Multiple / division / power / root: ] Polar / Euler form
\end{description}


\subsection{Calculations}
\begin{itemize}
	\item $z_1z_2=r_1r_2\cis(\theta_1+\theta_2)=r_1r_2e^{i((\theta_1+\theta_2))}$
	\item $\dfrac{z_1}{z_2}=\dfrac{r_1}{r_2}\cis(\theta_1-\theta_2)=\dfrac{r_1}{r_2}e^{i((\theta_1-\theta_2))}$
	\item $z^n = r^n\cis(n\theta)=r^ne^{i\theta n}$
	\item De Moivre's theorem: $\sqrt[n]{z}=\sqrt[n]{r}\cis(\dfrac{\theta+2k\pi}{n})=\sqrt[n]{r}e^{i(\dfrac{\theta+2k\pi}{n})}$ ($k=0,1,2,\dots,n-2,n-1$)
	\item $\sqrt{a+ib}=\pm(\sqrt{\dfrac{|z|+a}{2}}+i\dfrac{b}{|b|}\sqrt{\dfrac{|z|-a}{2}})$
	\item $\arg(z_1z_2)=\arg(z_1)+\arg(z_2)$
	\item $\arg(\dfrac{z_1}{z_2})=\arg(z_1)-\arg(z_2)$
	\item[$\star$] Complex numbers' sizes cannot be compared
\end{itemize}


\subsection{Conjugate}
\begin{itemize}
	\item $|z|=|z^*|$
	\item $z\cdot z^*= |z|^2$
	\item If $z=z*$ then z is a real number
	\item If $z=-z*$ then z is a pure imaginary number
\end{itemize}

\subsection{Deducing multiple angle formulae by complex number}
\subsubsection{Properties for $z$}
\begin{itemize}
	\item $z=\cis\theta=e^{i\theta}=\cos\theta + i\sin\theta$, $\dfrac{1}{z} = e^{-i\theta} = \cis (-\theta)$
	\item $z+\dfrac{1}{z}=2\cos\theta$
	\item $(z+\dfrac{1}{z})^n=(e^{i\theta}+e^{-i\theta})^n=2^n\cos^n\theta$
	\item $z-\dfrac{1}{z}=2i\sin\theta$
	\item $(z-\dfrac{1}{z})^n=(e^{i\theta}-e^{-i\theta})^n=2^n i^n \sin^n\theta$
	\item $z^n+\dfrac{1}{z^n}=2\cos (n\theta)$
	\item $z^n-\dfrac{1}{z^n}=2i\sin (n\theta)$
\end{itemize}
\subsubsection{Finding $\sin (n\theta)$ or $\cos (n\theta)$}
\begin{enumerate}
	\item Use $(z+\dfrac{1}{z})$ for $\cos$ and $(z+\dfrac{1}{z})$ for $\sin$
	\item Use the $n$th power of $z+\dfrac{1}{z}$ or $z-\dfrac{1}{z}$ ($n$ is the same as coefficient)
	\item Binomial expansion
	\item Merge the terms into $z^n+\dfrac{1}{z^n}$ or $z^n-\dfrac{1}{z^n}$
	\item Use properties above to convert then into trig functions
\end{enumerate}
\subsubsection{Find $\sin^n\theta$}
\textbf{$n$ is odd:}
\begin{itemize}
	\item $n=4k+1$: $\sin^n\theta=\dfrac{1}{2^{n-1}}[\binom{n}{0}\sin n\theta - \binom{n}{1}\sin (n-2)\theta + \binom{n}{1}\sin (n-4)\theta - \dots + \binom{n}{\frac{n-1}{2}}\sin \theta]$
	\item $n=4k+3$: $\sin^n\theta=\dfrac{1}{2^{n-1}}[-\binom{n}{0}\sin n\theta + \binom{n}{1}\sin (n-2)\theta - \binom{n}{1}\sin (n-4)\theta + \dots - \binom{n}{\frac{n-1}{2}}\sin \theta]$
\end{itemize}
\textbf{$n$ is even:}
\begin{itemize}
	\item $n=4k$: $\sin^n\theta=\dfrac{1}{2^{n-1}}[-\binom{n}{0}\cos n\theta + \binom{n}{1}\cos (n-2)\theta - \binom{n}{1}\cos (n-4)\theta + \dots - \binom{n}{\frac{n-1}{2}}\cos \theta]$
	\item $n=4k+2$: $\sin^n\theta=\dfrac{1}{2^{n-1}}[-\binom{n}{0}\cos n\theta + \binom{n}{1}\cos (n-2)\theta - \binom{n}{1}\cos (n-4)\theta + \dots - \binom{n}{\frac{n-1}{2}}\cos \theta]$
\end{itemize}
\subsubsection{Find $\cos^n\theta$}
\begin{itemize}
	\item $\cos^n\theta = \dfrac{1}{2^{n-1}}\sum$
	\item $\int \cos^n\theta \: d\theta = \dfrac{1}{2^{n-1}} \sum_{i=0}^{\ceil{\frac{n-1}{2}}} \binom{n}{i} \dfrac{\sin (n-2i)\theta}{den}$
\end{itemize}


\subsection{Argand diagrams}
\subsubsection{Circle}
\begin{itemize}
	\item $|z| = a$: circle with centre $O$, radius $a$
	\item $|z-z_0|=a$: circle with centre $z_0$, radius $a$
	\item $|z-z_0|<a$: circle with centre $z_0$, radius $a$, shaded in
	\item $|z-z_0|>a$: circle with centre $z_0$, radius $a$, outside of circle shaded
\end{itemize}
\subsubsection{Perpendicular bisector}
\begin{itemize}
	\item $|z-z_1|=|z-z_2|$: $z$ on perpendicular bisector of $z_1z_2$
	\item $|z-z_1|<|z-z_2|$ / $|z-z_1|>|z-z_2|$: on the left / right side of perpendicular bisector, bisector = dotted line
\end{itemize}
\subsubsection{Argument angle}
\begin{itemize}
	\item $\arg z = \theta$: ray starting from origin, angle with x-axis = $\theta$
	\item $\arg (z-z_0) = \theta$: ray starting from $z_0$, angle with horizontal = $\theta$
\end{itemize}
\subsubsection{Ellipse}
\begin{itemize}
	\item $|z-z_1|+|z-z_2|=2a$: ellipse with focus at $z_1$ and $z_2$, longest diameter = $2a$
\end{itemize}



