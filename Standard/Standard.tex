\documentclass[oneside,fleqn,11pt]{book}
\usepackage[a4paper, total={7.2in, 10.3in}]{geometry}
\usepackage{tikz}
\usetikzlibrary{calc}
\usepackage{setspace}
\usepackage{graphicx}
\usepackage{amsmath}
\usepackage{amssymb}
\DeclareMathOperator\dx{\mathrm{d}\mathit{x}}
\DeclareMathOperator\cis{cis}
\DeclareMathOperator\sech{sech}
\DeclareMathOperator\csch{csch}
\DeclareMathOperator\arsinh{arsinh}
\DeclareMathOperator\arcosh{arcosh}
\DeclareMathOperator\artanh{artanh}
\DeclareMathOperator\cosec{cosec}
\DeclareMathOperator\Nset{\mathbb{N}}
\DeclareMathOperator\Zset{\mathbb{Z}}
\DeclareMathOperator\Qset{\mathbb{Q}}
\DeclareMathOperator\Rset{\mathbb{R}}
\DeclareMathOperator\Iset{\mathbb{I}}
\DeclareMathOperator\Cset{\mathbb{C}}
\DeclareMathOperator\UQ{Q_3}
\DeclareMathOperator\LQ{Q_1}
\DeclareMathOperator\Med{Q_2}

\usepackage{pgfplots}
\graphicspath{ {./images/} }
\usepackage{bookmark}
\setcounter{tocdepth}{0}
\usepackage{import}
\usepackage{mathtools}

\makeatletter
\g@addto@macro\bfseries{\boldmath}
\makeatother

\DeclarePairedDelimiter{\ceil}{\lceil}{\rceil}
\hypersetup{
	colorlinks   = true, %Colours links instead of ugly boxes
	urlcolor     = blue, %Colour for external hyperlinks
	linkcolor    = black, %Colour of internal links
	citecolor   = red %Colour of citations
}

\newcommand{\tikzAngleOfLine}{\tikz@AngleOfLine}
\def\tikz@AngleOfLine(#1)(#2)#3{%
	\pgfmathanglebetweenpoints{%
		\pgfpointanchor{#1}{center}}{%
		\pgfpointanchor{#2}{center}}
	\pgfmathsetmacro{#3}{\pgfmathresult}%
}

% math font
\usepackage{amsmath}
\usepackage{amssymb}
\usepackage{amsthm}
\usepackage{mathtools}
%\usepackage{arev}

% color
\usepackage[table]{xcolor}

\usepackage[many]{tcolorbox}
\usepackage{pifont}
\usepackage{hyperref}
\usepackage{blindtext}
\counterwithin*{chapter}{part}
\newcommand*{\Part}[2][\partheading]{%
  \refstepcounter{part}%
  \def\partheading{#2}%
  \part*{#2}%
  \addcontentsline{toc}{part}{#1}%
}
% \hypersetup{%
% 	linktoc=all,%
% 	bookmarksnumbered,%
% 	bookmarksopen,%
% 	hidelinks}
% \usepackage{bookmark}
% \bookmarksetup{
% 	addtohook={%
% 		\ifnum\bookmarkget{level}=0%
% 		\bookmarksetup{color=red}%
% 		\fi%
% 		\ifnum\bookmarkget{level}=1%
% 		\bookmarksetup{color=blue}%
% 		\fi%
% 		\ifnum\bookmarkget{level}=2%
% 		\bookmarksetup{color=teal}%
% 		\fi}}
% % enumerate
\usepackage[inline]{enumitem}
\usepackage{multicol}
\usepackage[inline]{enumitem}
\usepackage{tasks}
\usepackage{caption}
\usepackage{subcaption}
\usepackage{cancel}

\renewcommand\rmdefault{ptm}
\renewcommand\sfdefault{ptm}

\tcbset{
	colframe=magenta,
	colback=magenta!12!white,
	boxed title style={colback=magenta},
	breakable,
	enhanced,
	sharp corners,
	boxsep=1pt,
	attach boxed title to top left={yshift=-\tcboxedtitleheight,  yshifttext=-.75\baselineskip},
	boxed title style={boxsep=1pt,sharp corners},
	fonttitle=\bfseries\sffamily,
	drop lifted shadow
}
\newtcolorbox{solution}[1][]{
	no shadow,
	top=2ex,
	boxrule=0pt,
	leftrule=1.4pt,
	title={Solution},
	colframe=green!79!blue,
	colback=green!12!white,
	boxed title style={colback=green!79!blue},
	overlay unbroken and first={
		\node[below right,font=\small,color=magenta,text width=.8\linewidth]
		at (title.north east) {#1};
	}
}
\newtcolorbox[auto counter,number within=chapter,number format=\arabic]{activity}[1][]{
	title={Activity~\thetcbcounter},
	colframe=green,
	colback=green!22!white,
	coltitle=black,
	boxed title style={colback=green},
	overlay unbroken and first={
		\node[below right,font=\small,color=green,text width=.8\linewidth]
		at (title.north east) {#1};
	}
}
\newtcolorbox[auto counter,number within=chapter,number format=\arabic]{definition}[1][]{
	title={Definition~\thetcbcounter},
	colframe=blue,
	colback=blue!12!white,
	boxed title style={colback=blue},
	overlay unbroken and first={
		\node[below right,font=\small,color=blue,text width=.8\linewidth]
		at (title.north east) {#1};
	}
}
\newtcolorbox[auto counter,number within=chapter,number format=\arabic]{theorem}[1][]{
	title={Theorem~\thetcbcounter},
	colframe=violet,
	colback=violet!12!white,
	fontupper=\itshape,
	boxed title style={colback=violet},
	overlay unbroken and first={
		\node[below right,font=\small,color=violet,text width=.8\linewidth]
		at (title.north east) {#1};
	}
}
\newtcolorbox[auto counter,number within=chapter,number format=\arabic]{example}[1][]{
	title={Example~\thetcbcounter},
	colframe=magenta,
	colback=magenta!12!white,
	boxed title style={colback=magenta},
	overlay unbroken and first={
		\node[below right,font=\small,color=magenta,text width=.8\linewidth]
		at (title.north east) {#1};
	}
}
\newtcolorbox[auto counter,number within=chapter,number format=\arabic]{exercise}[1][]{
	title={Exercise~\thetcbcounter},
	colframe=red,
	colback=red!12!white,
	boxed title style={colback=red},
	overlay unbroken and first={
		\node[below right,font=\small,color=red,text width=.8\linewidth]
		at (title.north east) {#1};
	}
}
\newtcolorbox[auto counter,number within=chapter,number format=\arabic]{generality}[1][]{
	title={Generality~\thetcbcounter},
	colframe=teal,
	colback=teal!12!white,
	boxed title style={colback=teal},
	overlay unbroken and first={
		\node[below right,font=\small,color=teal,text width=.8\linewidth]
		at (title.north east) {#1};
	}
}
\newtcolorbox[auto counter,number within=chapter,number format=\arabic]{property}[1][]{
	title={Property~\thetcbcounter},
	colframe=teal,
	colback=teal!12!white,
	boxed title style={colback=teal},
	overlay unbroken and first={
		\node[below right,font=\small,color=teal,text width=.8\linewidth]
		at (title.north east) {#1};
	}
}
\newtcolorbox{remark}[1][]{
	title={\scalebox{1.75}{\raisebox{-.25ex}{\ding{43}}}~Remark},
	colframe=yellow!45!white,
	colback=yellow!45!white,
	coltitle=violet,
	fontupper=\sffamily,
	boxed title style={colback=yellow!45!white},
	boxed title style={boxsep=1ex,sharp corners},%%
	overlay unbroken and first={
		\node[below right,font=\normalsize,color=red,text width=.8\linewidth]
		at (title.north east) {#1};
	}
}
\newtcolorbox{note}[1][]{
	title={\scalebox{1.75}{\raisebox{-0.25ex}{\ding{45}}}~Note},
	colframe=yellow!45!white,
	colback=yellow!45!white,
	coltitle=violet,
	fonttitle=\bfseries\sffamily,
	fontupper=\sffamily,
	boxed title style={colback=yellow!45!white},
	boxed title style={boxsep=1ex,sharp corners},%%
	overlay unbroken and first={
		\node[below right,font=\normalsize,color=red,text width=.8\linewidth]
		at (title.north east) {#1};
	}
}

\makeatletter
\g@addto@macro\bfseries{\boldmath}
\makeatother

\title{A Level Pure Maths Notes}
\author{Xingzhi Lu}
\date{}

\begin{document}
\everymath{\displaystyle}
\maketitle
\tableofcontents

\part{Pure 1}
\chapter{Algebraic expressions}
\section{Finding areas of shapes}
\begin{itemize}
    \item Area of triangle $\bigtriangleup ABC$ = $\dfrac{1}{2}|\overrightarrow{AB}\times\overrightarrow{AC}|$
    \item Area of parallelogram $ABCD$ = $|(b-a) \times (d-a)|=|(a\times b)+(b \times d) + (d \times a)|$ ($A$, $B$, $C$, $D$ have position vector $a$, $b$, $c$, $d$ respectively)
\end{itemize}
\section{Scalar triple product}
\begin{itemize}
    \item Volume of parallelepiped = $a\cdot(b\times c)$ ($a$, $b$, $c$ = 3 different sides)
    \item Volume of tetrahedron $ABCD$ = $\dfrac{1}{6}|\overrightarrow{AD}\cdot(\overrightarrow{AB}\times\overrightarrow{AC})|$
\end{itemize}
\section{Straight lines}
\subsection{Vector equation of line}
\begin{itemize}
    \item $(\vec{r}-\vec{a})\times\vec{b}=0$
    \item $\vec{a}$ = position vector of a point on line, $\vec{b}$ = directional vector
\end{itemize}
\subsection{Direction cosines}
For straight line $(r-a)\times b=0$, where $a=x\vec{i}+y\vec{j}+z\vec{k}$ and the line makes angle $\alpha$, $\beta$ and $\gamma$ with the positive $x$-, $y$- and $z$-axes respectively:
\begin{itemize}
    \item $l=\cos\alpha = \dfrac{x}{|a|}$
    \item $m=\cos\beta = \dfrac{y}{|a|}$
    \item $n=\cos\gamma = \dfrac{z}{|a|}$
    \item $l^2+m^2+n^2=1$
\end{itemize}

\chapter{Quadratics}
% \section{Circle}
% \subsection{Definition of circle}
% \subsubsection{Cartesian form}
% \begin{itemize}
%     \item $(x-a)^2+(y-b)^2=r^2$
% \end{itemize}

% \subsubsection{Parametric form}
% \begin{itemize}
%     \item $x=a+r\cos\theta$
%     \item $y=b+r\sin\theta$
% \end{itemize}

% \subsubsection{Polar form}
% \begin{itemize}
%     \item $r=a$
%     \item $r=a\sin\theta$
%     \item $r=a\cos\theta$
% \end{itemize}

\section{Parabola}
\subsection{Graph}
\includegraphics[width=0.6\textwidth]{parabola.png}
\begin{itemize}
    \item Symmetrix about the $x$-axis
    \item Focus at $(a, 0)$
    \item Vertex at $(0, 0)$
\end{itemize}
\subsection{Definition}
\begin{itemize}
    \item The locus of points that are the \textbf{same distance} from a fixed point, $S$,
          called the \textbf{focus}, and a fixed straight line called the \textbf{directrix}
    \item $\dfrac{\text{distance to foci}}{\text{distance to directrix}} = e = 1$
\end{itemize}
\subsection{Cartesian equation}
\begin{itemize}
    \item $y^2=4ax$ ($a>0$)
\end{itemize}
\subsection{Parametric equation}
\begin{itemize}
    \item $x=at^2$
    \item $y=2at$
    \item $t\in \Rset$
\end{itemize}
\subsection{Eccentricity}
\begin{itemize}
    \item $e=1$
\end{itemize}
\subsection{Directrix}
\begin{itemize}
    \item The directrix has equation $x+a=0$
\end{itemize}
\subsection{Tangents and normals}
\begin{itemize}
    \item $\dfrac{\dy}{\dx} = \frac{1}{t} = \frac{2a}{y}$
    \item Equation of tangent: $ty=x+at^2$ at $P(at^2, 2at)$
    \item Equation of normal: $y+tx=2at+at^3$ at $P(at^2, 2at)$
\end{itemize}

\section{Rectangular hyperbolas}
\subsection{Graph}
\includegraphics[width=0.3\textwidth]{rectangular_hyperbola.png}
\begin{itemize}
    \item Asymptotes at $x=0$ and $y=0$ ($x$ and $y$-axis)
\end{itemize}
\subsection{Definition}
\begin{itemize}
    \item The locus of points that are the \textbf{same distance} from a fixed point, $S$,
          called the \textbf{focus}, and a fixed straight line called the \textbf{directrix}
    \item $\dfrac{\text{distance to foci}}{\text{distance to directrix}} = e = 1$
\end{itemize}
\subsection{Cartesian equation}
\begin{itemize}
    \item $xy=c^2$ ($c>0$)
\end{itemize}
\subsection{Parametric equation}
\begin{itemize}
    \item $x=ct$
    \item $y=\frac{c}{t}$
    \item $t \neq 0, t\in\Rset$
\end{itemize}
\subsection{Eccentricity}
\begin{itemize}
    \item $e=\sqrt{2}$
\end{itemize}
\subsection{Directrix}
\begin{itemize}
    \item $x+y=\pm c\sqrt{2}$
\end{itemize}
\subsection{Tangents and normals}
\begin{itemize}
    \item Equation of tangent: $x+t^2y=2ct$ at $P(ct, \frac{c}{t})$
    \item Equation of normal: $t^3x-ty=c(t^4-1)$ at $P(ct, \frac{c}{t})$
\end{itemize}

\chapter{Equations and inequalities}
\section{Hooke's law}
\begin{itemize}
    \item $\text{Tension produced}\propto x$ $\rightarrow$ $T=kx$, where $k$ is a constant
    \item $k$ depends on the unstretched length of the string or spring ($l$) and the \textbf{modulus of elasticity of the string or spring} ($\lambda$, unit = N)
    \item Hence $T=\frac{\lambda x}{l}$
    \item[*] Can also be applied if the string or spring is compressed
\end{itemize}

\section{Elastic energy}
\begin{itemize}
    \item Work done in stretching an elastic string or spring of modulus of elasticity $\lambda$ from its natural length $l$ to a length of $(l+x)$ = $\frac{\lambda x^2}{2l}$
    \item Elastic potential energy stored = amount of energy used to stretch the spring to a length of $(l+x)$ = $\frac{1}{2}kx^2$ = $\frac{\lambda x^2}{2l}$
    \item[*] Can also be applied when an elastic string or spring is compressed
\end{itemize}

\chapter{Graphs and transformations}
\section{Vieta's Law}
For $a_{n}x^{n}+a_{n-1}x^{n-1}+\cdots +a_{1}x+a_{0}=0$:
\begin{itemize}
    \item $\sum x_i=-\frac{a_{n-1}}{a_n}$
    \item $\sum x_ix_j=\frac{a_{n-2}}{a_n}$
    \item $\sum x_ix_jx_k = -\frac{a_{n-3}}{a_n}$
    \item $\sum _{1\leq i_{1}<i_{2}<\cdots <i_{k}\leq n}\left(\prod _{j=1}^{k}r_{i_{j}}\right)=(-1)^{k}{\frac {a_{n-k}}{a_{n}}}$
    \item $\prod x_i=(-1)^n\frac{a_0}{a_n}$
\end{itemize}

\chapter{Straight line graphs}
\section{Resolving forces}
\begin{enumerate}
    \item Resolve perpendicular and parallel to the direction of motion
    \item Clearly indicate the direction of motion
    \item Use $\sum F = ma$
\end{enumerate}

\section{Friction}
\begin{itemize}
    \item $F_r \leq \mu R$
    \item Friction is only \textbf{as large as it needs to be} to oppose the motion
    \item If the object is moving then $F_r = \mu R$
\end{itemize}



\chapter{Circles}
\section{Basic calculations}
\subsection{Addition / subtraction}
$(\mathbf {A}+\mathbf {B})_{i,j}=\mathbf {A}_{i,j}+{\mathbf {B}}_{i,j},\quad 1\leq i\leq m,\quad 1\leq j\leq n$\\
$(\mathbf {A}-\mathbf {B})_{i,j}=\mathbf {A}_{i,j}-{\mathbf {B}}_{i,j},\quad 1\leq i\leq m,\quad 1\leq j\leq n$
\subsection{Scalar multiplication}
$(c\mathbf{A})_{i,j}=c\cdot A_{i,j}$
\subsection{Matrix multiplication}
\includegraphics[width=0.35\textwidth]{MatrixMultiplication}

\subsection{Transposition}
$(\mathbf{A}^\mathrm{T})_{i,j}=A_{j,i}$

\section{Special matrices}
\begin{description}
	\item[Square matrix:] The number of rows and columns are the same
	\item[Zero matrix:] All of the elements are zero
	\item[Identity matrix:] A square matrix in which all the elements on the leading diagonal are 1 and the remaining elements are 0, denoted by $\mathbf{I}_k$ for $k\times k$ identity matrix
\end{description}


\section{Determinants}

\subsection{$2\times2$ matrices}
$\begin{vmatrix}a&b\\c&d\end{vmatrix}=ad-bc$

\subsection{$3\times3$ matrices}

$\begin{vmatrix}a&b&c\\d&e&f\\g&h&i\end{vmatrix}=a\begin{vmatrix}e&f\\h&i\end{vmatrix}-b\begin{vmatrix}d&f\\g&i\end{vmatrix}+c\begin{vmatrix}d&e\\g&h\end{vmatrix}=aei+bfg+cdh-ceg-bdi-afh$

\subsection{Singular matrices}
\begin{itemize}
	\item Singular matrices are square matrices with a determinant of 0
	\item It does not have an inverse
	\item If $\mathbf{A}$ and $\mathbf{B}$ are non-singular matrices, then $(\mathbf{AB})^{-1}=\mathbf{B}^{-1}\mathbf{A}^{-1}$
\end{itemize}



\subsection{Properties of determinants}
\begin{itemize}
	\item $\det(\mathbf{AB})=\det(\mathbf{A})\det(\mathbf{B})=\det(\mathbf{B})\det(\mathbf{A})=\det(\mathbf{BA})$
	\item $\det(k\mathbf{A})=k^n\det(\mathbf{A})$ ($\mathbf{A}$ is a $n\times n$ matrix)
\end{itemize}


\section{Inverse matrices}
\subsection{$2\times2$ matrices}
$\begin{bmatrix}
	a & b\\c & d
\end{bmatrix}^{-1}=\dfrac{1}{ad-bc}\begin{bmatrix}
	d & -b\\-c & a
\end{bmatrix}$

\subsection{$3\times3$ matrices}
$\mathbf {A} ^{-1}={\begin{bmatrix}a&b&c\\d&e&f\\g&h&i\\\end{bmatrix}}^{-1}={\frac {1}{\det(\mathbf {A} )}}{\begin{bmatrix}\,A&\,B&\,C\\\,D&\,E&\,F\\\,G&\,H&\,I\\\end{bmatrix}}^{\mathrm {T} }={\frac {1}{\det(\mathbf {A} )}}{\begin{bmatrix}\,A&\,D&\,G\\\,B&\,E&\,H\\\,C&\,F&\,I\\\end{bmatrix}}$\\
$\begin{alignedat}{6}A&={}&(ei-fh),&\quad &D&={}&-(bi-ch),&\quad &G&={}&(bf-ce),\\B&={}&-(di-fg),&\quad &E&={}&(ai-cg),&\quad &H&={}&-(af-cd),\\C&={}&(dh-eg),&\quad &F&={}&-(ah-bg),&\quad &I&={}&(ae-bd).\\\end{alignedat}$

\subsection{Solving equations with matrices}
If $\mathbf{A}\begin{pmatrix}
	x\\y\\z
\end{pmatrix}=\mathbf{v}$ then $\begin{pmatrix}
	x\\y\\z
\end{pmatrix}=\mathbf{A}^{-1}\mathbf{v}$
\begin{description}
	\item[Consistent system of linear equations:] there is at least one set of values that satisfies all the equations simultaneously
	\item[Inconsistent:] such set of values does not exist
\end{description}

\subsection{Possible outcomes of solutions}
\includegraphics[width=0.75\textwidth]{equations}\\
\includegraphics[width=0.5\textwidth]{equations2}

\chapter{Algebraic methods}
\subsection{Polar form}
\begin{itemize}
	\item $P=(r,\theta)$
	\item $r=f(\theta)$
\end{itemize}

\subsection{Converting between different forms of coordinates}
\subsubsection{Polar to Cartesian form}
\begin{itemize}
	\item $x=r\cos\theta$
	\item $y=r\sin\theta$
\end{itemize}
\subsubsection{Cartesian to polar form}
\begin{itemize}
	\item $r=\sqrt{x^2+y^2}$
	\item $\tan\theta=\dfrac{y}{x}$
\end{itemize}


\subsection{Sketching curves for polar equations}
\begin{enumerate}
	\item Plot points
	\item Consider symmetry
	\item Convert to Cartesian form
\end{enumerate}
\subsubsection{Circle polar form}
\begin{itemize}
	\item 
\end{itemize}
\subsubsection{Polar form of lines}
\subsubsection{Cardioids}
\subsubsection{Lima con}
\subsubsection{Rose}
\subsubsection{Lemniscates}
\subsubsection{Spiral}

\subsection{Testing for symmetry}


\subsection{Finding area enclosed}
Area = $\dfrac{1}{2}\int_{\alpha}^{\beta} r^2d\theta$

\chapter{The binomial expansion}
\section{Hyperbolic function definitions}
\subsection{$\sinh x$}
\begin{description}
	\item[Definition] $\sinh x = \dfrac{e^x-e^{-x}}{2}$
	\item[Domain] $x \in \Rset$
	\item[Range] $\sinh x \in \Rset$
	\item[Asymptotes] $x\rightarrow +\infty$, $y\rightarrow\dfrac{e^x}{2}$; $x\rightarrow -\infty$, $y\rightarrow -\dfrac{e^{-x}}{2}$
	\item[x-intercept] $(0,0)$
	\item[y-intercept] $(0,0)$	
\end{description}

\subsection{$\cosh x$}
\begin{description}
	\item[Definition] $\cosh x = \dfrac{e^x+e^{-x}}{2}$
	\item[Domain] $x \in \Rset$
	\item[Range] $\cosh x \geq 1$
	\item[Asymptotes] $x\rightarrow +\infty$, $y\rightarrow\dfrac{e^x}{2}$; $x\rightarrow -\infty$, $y\rightarrow\dfrac{e^{-x}}{2}$
	\item[x-intercept] No
	\item[y-intercept] $(0,1)$
\end{description}

\subsection{$\tanh x$}
\begin{description}
	\item[Definition] $\tanh x = \dfrac{\sinh x}{\cosh x}=\dfrac{e^x-e^{-x}}{e^x+e^{-x}}$
	\item[Domain] $x \in \textbf{R}$
	\item[Range] $-1 < \tanh x < 1$
	\item[Asymptotes] $x\rightarrow +\infty$, $y\rightarrow 1$; $x\rightarrow -\infty$, $y\rightarrow -1$
	\item[x-intercept] $(0,0)$
	\item[y-intercept] $(0,0)$
\end{description}

\subsection{Function graphs}
\includegraphics[width=\linewidth]{images/hyperbolic_graphs}

\section{Inverse hyperbolic functions}
\begin{itemize}
	\item $\arsinh x = \ln \left[x+\sqrt{x^2+1}\right]$
	\item $\arcosh x = \ln \left[x+\sqrt{x^2-1}\right] \:\: (x \geq 1)$
	\item $\artanh x = \ln \left[\dfrac{1+x}{1-x}\right] \:\: (-1 < x < 1)$
\end{itemize}

\subsection{Proof}
\begin{example}
	Show that $\arsinh x = \ln \left[x+\sqrt{x^2+1}\right]$
\end{example}

\begin{solution}
	Let $y=\arsinh x$
	\begin{align*}
		\sinh y &= x\\
		\dfrac{e^y-e^{-y}}{2} &= x\\
		e^y-e{-y} &= 2x\\
		e^{2y} - 1 &= 2x e^y\\
		(e^y-x)^2 &= x^2 + 1\\
		e^y &= x \pm \sqrt{x^2 + 1}
	\end{align*}
	$e^y = x + \sqrt{x^2 + 1}$ \text{since $\sqrt{x^2 + 1} > 0$ so it makes $e^y$ negative which is impossible}\\
	Hence $y = \ln \left[x+\sqrt{x^2+1}\right]$ so $\arsinh x = \ln \left[x+\sqrt{x^2+1}\right]$
\end{solution}

\begin{remark}
	Prove these identities by finding value of $e^y$ in quadratic equations, think about domain when deciding the sign before the square root
\end{remark}

\subsection{Graphs}
\includegraphics[width=\linewidth]{images/hyperbolic_inverse_graphs}



\section{Identities of hyperbolic functions}
Similar to trigonometric identities:
\begin{itemize}
	\item $\tanh x = \dfrac{\sinh x}{\cosh x}$
	\item $\cosh^2 x - \sinh^2 x = 1$
	\item $\tanh^2 x + \sech^2 x= 1$
	\item $\coth^2 x - \csch^2 x = 1$
\end{itemize}

\subsection{Addition}
\begin{itemize}
	\item $\sinh(x+y)=\sinh x \cosh y + \sinh y \cosh x$
	\item $\cosh(x+y)=\cosh x \cosh y + \sinh x \sinh y$
	\item $\tanh(x+y) = \dfrac{\sinh(x+y)}{\cosh(x+y)} = \dfrac{\sinh x \cosh y + \sinh y \cosh x}{\cosh x \cosh y + \sinh x \sinh y} = \dfrac{\frac{\sinh x}{\cosh x}+\frac{\sinh y}{\cosh y}}{1 + \frac{\sinh x \sinh y}{\cosh x \cosh y}}=\dfrac{\tanh x + \tanh y}{1 + \tanh x \tanh y}$
\end{itemize}

\subsection{Double angle}
\begin{itemize}
	\item $\sinh 2x = 2\sinh x \cosh x$
	\item $\cosh 2x = \cosh^2 x + \sinh^2 x = 2\cosh^2 x - 1 = 2\sinh^2 + 1$
	\item $\tanh 2x = \dfrac{2\tanh x}{1 + \tanh^2 x}$
\end{itemize}

\subsection{Power descending}
\begin{itemize}
	\item $\sinh^2 x = \dfrac{\cosh 2x - 1}{2}$
	\item $\cosh^2 x = \dfrac{\cosh 2x + 1}{2}$
\end{itemize}


\section{Calculus with hyperbolic functions}
\subsection{Differentiation}
\[(\sinh x)'=\cosh x\]
\[(\cosh x)'=\sinh x\]
\[(\tanh x)'=1-\tanh^2 x = \sech^2 x\]
\[(\csch x)'=-\coth x \csch x\]
\[(\sech x)'=-\sech x \tanh x\]
\[(\coth x)'=-\csch^2 x\]
\[(\arsinh x)' = \dfrac{1}{\sqrt{1+x^2}}\]
\[(\arcosh x)' = \dfrac{1}{\sqrt{x^2-1}}\]
\[(\artanh x)' = \dfrac{1}{1-x^2}\]


\subsection{Integration}
\[\int\sinh x \dx = \cosh x + c\]
\[\int\cosh x \dx = \sinh x + c\]
\[\int \tanh x \dx = \ln |\cosh x| + c\]
\[\int \coth x \dx = \ln |\sinh x| + c\]
\[\int \sech x \dx = \ln |-\sech x + \tanh x| + c = \ln \left|\tan\left(\frac{1}{2}x+\frac{1}{4}\pi \right)\right| + c\]
\[\int \csch x \dx = -\ln|\csch x + \coth x| + c\]
\[\int \sinh^2 x \dx = \int \frac{\cosh 2x - 1}{2} \dx = \frac{\sinh 2x - 2x}{4} + c\]
\[\int \cosh^2 x \dx = \int \frac{\cosh 2x + 1}{2} \dx = \frac{\sinh 2x + 2x}{4} + c\]
\[\int \tanh^2 x \dx = \int 1-\sech^2 x \dx = x - \tanh x + c\]
\[\int \coth^2 x \dx = \int 1+\csch^2 x \dx = x - \coth x + c\]
\[\int \sinh x \cosh x \dx = \int \frac{\sinh 2x}{2} \dx = \dfrac{\cosh 2x}{4} + c\]
\[\int \tanh x \sech x \dx = -\sech x + c\]
\[\int \coth x \csch x \dx = -\csch x + c\]
\[\int \frac{1}{\sqrt{x^2-a^2}} \dx= \arcosh \left(\frac{x}{a}\right) + c = \ln \left(x + \sqrt{x^2-a^2}\right) + c \:\:  (x>a)\]
\[\int \frac{1}{\sqrt{a^2+x^2}} \dx= \arsinh \left(\frac{x}{a}\right) + c = \ln \left(x + \sqrt{x^2+a^2}\right) + c \]
\[\int \frac{1}{a^2-x^2} \dx=\frac{1}{a}\artanh \left(\frac{x}{a}\right) + c = \frac{1}{2a}\ln \left|\frac{a+x}{a-x}\right| + c\]


\chapter{Trigonometric ratios}
\section{SUVAT equations}
\begin{itemize}
	\item $s=ut+\dfrac{1}{2}at^2$
	\item $s=vt-\dfrac{1}{2}at^2$
	\item $v=u+at$
	\item $v^2=u^2+2as$
	\item $s=\dfrac{1}{2}(u+v)t$
\end{itemize}


\chapter{Trigonometric identities and equations}
\section{Locating roots}
\subsection{Method}
If a function $f(x)$ is continuous on the interval $[a,b]$ and $f(a)$ and $f(b)$ have opposite signs, then $f(x)$ has at least one root, $x$, which satisfies $a<x<b$
\subsection{How change of sign can fail}
\begin{itemize}
    \item When the interval is too large sign may not change as there may be an even number of roots
    \item If the function is not continuous, sign may change but there may be an asymptote e.g. reciprocal graph
\end{itemize}

\subsection{Model answer}
\begin{itemize}
    \item $f\left( a \right) = \dots$
    \item $f\left( b \right) = \dots$
    \item There is a change of sign in the interval $[a, b]$ and $f\left( x \right)$ is continuous so there is at least one root in this interval
\end{itemize}

\section{Iteration diagrams}
\subsection{Convergent diagrams}
\includegraphics[width=\linewidth]{Cobwebstaircase.png}
\subsection{Divergent diagrams}
\includegraphics[width=0.6\linewidth]{fixed_point_iteration_im3.png}

\chapter{Vectors}
\section{Finding distance travelled}
\begin{itemize}
	\item Use graph to show sign changes during the interval
	\item Remember to account for periods with negative velocity
\end{itemize}

\chapter{Differentiation}
\section{The first principle}
$$f'(x) = \lim\limits_{n \to 0}\dfrac{f(x+h)-f(x)}{h}$$


\chapter{Integration}
\input{chapters/P1/13.tex}

\chapter{Exponentials and logarithms}
\section{Sketching graphs}
Find the y-intercept of the graph

\section{$e^x$ function}
\begin{description}
    \item $(e^x)'$ = $e^x$ (gradient = $y$ value)
    \item $(e^{kx})'$ = $ke^{kx}$ (gradient directly proportional to $y$ value)
\end{description}

\section{Logarithm}
$a^x = n$: $\log_a n = x$ ($a\neq1$ and $a>0$, $x>0$)
\subsection{Laws}
\begin{description}
    \item[The multiplication law:] $\log_a x + \log_a y = \log_a xy$
    \item[The division law:] $\log_a x - \log_a y = \log_a (\dfrac{x}{y})$
    \item[The power law:] $\log_a x^k = k\log_a x$
    \item[Change base formula]: $\log_a b = \dfrac{\log_c b}{\log_c a}$
\end{description}

\subsection{Logarithms in non-linear form}
\textbf{Exponential}
\begin{itemize}
    \item $y=ab^x\rightarrow\ln y = x\ln b + \ln a$
    \item x-axis = $x$, y-axis = $\ln y$, gradient = $\ln b$, y-intercept = $\ln a$
\end{itemize}
\textbf{Power}
\begin{itemize}
    \item $y=ax^b\rightarrow\ln y = b\ln x + \ln a$
    \item x-axis = $\ln x$, y-axis = $\ln y$, gradient = $b$, y-intercept = $\ln a$
\end{itemize}
\textbf{Logarithmic}
\begin{itemize}
    \item $y=a\ln x\rightarrow$ kept the same
    \item x-axis = $\ln x$, y-axis = $y$, gradient = $a$
\end{itemize}

\part{A2 Contents}
\chapter{Algebraic methods}
\section{Finding areas of shapes}
\begin{itemize}
    \item Area of triangle $\bigtriangleup ABC$ = $\dfrac{1}{2}|\overrightarrow{AB}\times\overrightarrow{AC}|$
    \item Area of parallelogram $ABCD$ = $|(b-a) \times (d-a)|=|(a\times b)+(b \times d) + (d \times a)|$ ($A$, $B$, $C$, $D$ have position vector $a$, $b$, $c$, $d$ respectively)
\end{itemize}
\section{Scalar triple product}
\begin{itemize}
    \item Volume of parallelepiped = $a\cdot(b\times c)$ ($a$, $b$, $c$ = 3 different sides)
    \item Volume of tetrahedron $ABCD$ = $\dfrac{1}{6}|\overrightarrow{AD}\cdot(\overrightarrow{AB}\times\overrightarrow{AC})|$
\end{itemize}
\section{Straight lines}
\subsection{Vector equation of line}
\begin{itemize}
    \item $(\vec{r}-\vec{a})\times\vec{b}=0$
    \item $\vec{a}$ = position vector of a point on line, $\vec{b}$ = directional vector
\end{itemize}
\subsection{Direction cosines}
For straight line $(r-a)\times b=0$, where $a=x\vec{i}+y\vec{j}+z\vec{k}$ and the line makes angle $\alpha$, $\beta$ and $\gamma$ with the positive $x$-, $y$- and $z$-axes respectively:
\begin{itemize}
    \item $l=\cos\alpha = \dfrac{x}{|a|}$
    \item $m=\cos\beta = \dfrac{y}{|a|}$
    \item $n=\cos\gamma = \dfrac{z}{|a|}$
    \item $l^2+m^2+n^2=1$
\end{itemize}

\chapter{Functions and graphs}
% \section{Circle}
% \subsection{Definition of circle}
% \subsubsection{Cartesian form}
% \begin{itemize}
%     \item $(x-a)^2+(y-b)^2=r^2$
% \end{itemize}

% \subsubsection{Parametric form}
% \begin{itemize}
%     \item $x=a+r\cos\theta$
%     \item $y=b+r\sin\theta$
% \end{itemize}

% \subsubsection{Polar form}
% \begin{itemize}
%     \item $r=a$
%     \item $r=a\sin\theta$
%     \item $r=a\cos\theta$
% \end{itemize}

\section{Parabola}
\subsection{Graph}
\includegraphics[width=0.6\textwidth]{parabola.png}
\begin{itemize}
    \item Symmetrix about the $x$-axis
    \item Focus at $(a, 0)$
    \item Vertex at $(0, 0)$
\end{itemize}
\subsection{Definition}
\begin{itemize}
    \item The locus of points that are the \textbf{same distance} from a fixed point, $S$,
          called the \textbf{focus}, and a fixed straight line called the \textbf{directrix}
    \item $\dfrac{\text{distance to foci}}{\text{distance to directrix}} = e = 1$
\end{itemize}
\subsection{Cartesian equation}
\begin{itemize}
    \item $y^2=4ax$ ($a>0$)
\end{itemize}
\subsection{Parametric equation}
\begin{itemize}
    \item $x=at^2$
    \item $y=2at$
    \item $t\in \Rset$
\end{itemize}
\subsection{Eccentricity}
\begin{itemize}
    \item $e=1$
\end{itemize}
\subsection{Directrix}
\begin{itemize}
    \item The directrix has equation $x+a=0$
\end{itemize}
\subsection{Tangents and normals}
\begin{itemize}
    \item $\dfrac{\dy}{\dx} = \frac{1}{t} = \frac{2a}{y}$
    \item Equation of tangent: $ty=x+at^2$ at $P(at^2, 2at)$
    \item Equation of normal: $y+tx=2at+at^3$ at $P(at^2, 2at)$
\end{itemize}

\section{Rectangular hyperbolas}
\subsection{Graph}
\includegraphics[width=0.3\textwidth]{rectangular_hyperbola.png}
\begin{itemize}
    \item Asymptotes at $x=0$ and $y=0$ ($x$ and $y$-axis)
\end{itemize}
\subsection{Definition}
\begin{itemize}
    \item The locus of points that are the \textbf{same distance} from a fixed point, $S$,
          called the \textbf{focus}, and a fixed straight line called the \textbf{directrix}
    \item $\dfrac{\text{distance to foci}}{\text{distance to directrix}} = e = 1$
\end{itemize}
\subsection{Cartesian equation}
\begin{itemize}
    \item $xy=c^2$ ($c>0$)
\end{itemize}
\subsection{Parametric equation}
\begin{itemize}
    \item $x=ct$
    \item $y=\frac{c}{t}$
    \item $t \neq 0, t\in\Rset$
\end{itemize}
\subsection{Eccentricity}
\begin{itemize}
    \item $e=\sqrt{2}$
\end{itemize}
\subsection{Directrix}
\begin{itemize}
    \item $x+y=\pm c\sqrt{2}$
\end{itemize}
\subsection{Tangents and normals}
\begin{itemize}
    \item Equation of tangent: $x+t^2y=2ct$ at $P(ct, \frac{c}{t})$
    \item Equation of normal: $t^3x-ty=c(t^4-1)$ at $P(ct, \frac{c}{t})$
\end{itemize}

\chapter{Sequences and series}
\section{Hooke's law}
\begin{itemize}
    \item $\text{Tension produced}\propto x$ $\rightarrow$ $T=kx$, where $k$ is a constant
    \item $k$ depends on the unstretched length of the string or spring ($l$) and the \textbf{modulus of elasticity of the string or spring} ($\lambda$, unit = N)
    \item Hence $T=\frac{\lambda x}{l}$
    \item[*] Can also be applied if the string or spring is compressed
\end{itemize}

\section{Elastic energy}
\begin{itemize}
    \item Work done in stretching an elastic string or spring of modulus of elasticity $\lambda$ from its natural length $l$ to a length of $(l+x)$ = $\frac{\lambda x^2}{2l}$
    \item Elastic potential energy stored = amount of energy used to stretch the spring to a length of $(l+x)$ = $\frac{1}{2}kx^2$ = $\frac{\lambda x^2}{2l}$
    \item[*] Can also be applied when an elastic string or spring is compressed
\end{itemize}

\chapter{Binomial expansion}
\section{Vieta's Law}
For $a_{n}x^{n}+a_{n-1}x^{n-1}+\cdots +a_{1}x+a_{0}=0$:
\begin{itemize}
    \item $\sum x_i=-\frac{a_{n-1}}{a_n}$
    \item $\sum x_ix_j=\frac{a_{n-2}}{a_n}$
    \item $\sum x_ix_jx_k = -\frac{a_{n-3}}{a_n}$
    \item $\sum _{1\leq i_{1}<i_{2}<\cdots <i_{k}\leq n}\left(\prod _{j=1}^{k}r_{i_{j}}\right)=(-1)^{k}{\frac {a_{n-k}}{a_{n}}}$
    \item $\prod x_i=(-1)^n\frac{a_0}{a_n}$
\end{itemize}

\chapter{Radians}
\section{Resolving forces}
\begin{enumerate}
    \item Resolve perpendicular and parallel to the direction of motion
    \item Clearly indicate the direction of motion
    \item Use $\sum F = ma$
\end{enumerate}

\section{Friction}
\begin{itemize}
    \item $F_r \leq \mu R$
    \item Friction is only \textbf{as large as it needs to be} to oppose the motion
    \item If the object is moving then $F_r = \mu R$
\end{itemize}



\chapter{Trigonometric functions}
\section{Basic calculations}
\subsection{Addition / subtraction}
$(\mathbf {A}+\mathbf {B})_{i,j}=\mathbf {A}_{i,j}+{\mathbf {B}}_{i,j},\quad 1\leq i\leq m,\quad 1\leq j\leq n$\\
$(\mathbf {A}-\mathbf {B})_{i,j}=\mathbf {A}_{i,j}-{\mathbf {B}}_{i,j},\quad 1\leq i\leq m,\quad 1\leq j\leq n$
\subsection{Scalar multiplication}
$(c\mathbf{A})_{i,j}=c\cdot A_{i,j}$
\subsection{Matrix multiplication}
\includegraphics[width=0.35\textwidth]{MatrixMultiplication}

\subsection{Transposition}
$(\mathbf{A}^\mathrm{T})_{i,j}=A_{j,i}$

\section{Special matrices}
\begin{description}
	\item[Square matrix:] The number of rows and columns are the same
	\item[Zero matrix:] All of the elements are zero
	\item[Identity matrix:] A square matrix in which all the elements on the leading diagonal are 1 and the remaining elements are 0, denoted by $\mathbf{I}_k$ for $k\times k$ identity matrix
\end{description}


\section{Determinants}

\subsection{$2\times2$ matrices}
$\begin{vmatrix}a&b\\c&d\end{vmatrix}=ad-bc$

\subsection{$3\times3$ matrices}

$\begin{vmatrix}a&b&c\\d&e&f\\g&h&i\end{vmatrix}=a\begin{vmatrix}e&f\\h&i\end{vmatrix}-b\begin{vmatrix}d&f\\g&i\end{vmatrix}+c\begin{vmatrix}d&e\\g&h\end{vmatrix}=aei+bfg+cdh-ceg-bdi-afh$

\subsection{Singular matrices}
\begin{itemize}
	\item Singular matrices are square matrices with a determinant of 0
	\item It does not have an inverse
	\item If $\mathbf{A}$ and $\mathbf{B}$ are non-singular matrices, then $(\mathbf{AB})^{-1}=\mathbf{B}^{-1}\mathbf{A}^{-1}$
\end{itemize}



\subsection{Properties of determinants}
\begin{itemize}
	\item $\det(\mathbf{AB})=\det(\mathbf{A})\det(\mathbf{B})=\det(\mathbf{B})\det(\mathbf{A})=\det(\mathbf{BA})$
	\item $\det(k\mathbf{A})=k^n\det(\mathbf{A})$ ($\mathbf{A}$ is a $n\times n$ matrix)
\end{itemize}


\section{Inverse matrices}
\subsection{$2\times2$ matrices}
$\begin{bmatrix}
	a & b\\c & d
\end{bmatrix}^{-1}=\dfrac{1}{ad-bc}\begin{bmatrix}
	d & -b\\-c & a
\end{bmatrix}$

\subsection{$3\times3$ matrices}
$\mathbf {A} ^{-1}={\begin{bmatrix}a&b&c\\d&e&f\\g&h&i\\\end{bmatrix}}^{-1}={\frac {1}{\det(\mathbf {A} )}}{\begin{bmatrix}\,A&\,B&\,C\\\,D&\,E&\,F\\\,G&\,H&\,I\\\end{bmatrix}}^{\mathrm {T} }={\frac {1}{\det(\mathbf {A} )}}{\begin{bmatrix}\,A&\,D&\,G\\\,B&\,E&\,H\\\,C&\,F&\,I\\\end{bmatrix}}$\\
$\begin{alignedat}{6}A&={}&(ei-fh),&\quad &D&={}&-(bi-ch),&\quad &G&={}&(bf-ce),\\B&={}&-(di-fg),&\quad &E&={}&(ai-cg),&\quad &H&={}&-(af-cd),\\C&={}&(dh-eg),&\quad &F&={}&-(ah-bg),&\quad &I&={}&(ae-bd).\\\end{alignedat}$

\subsection{Solving equations with matrices}
If $\mathbf{A}\begin{pmatrix}
	x\\y\\z
\end{pmatrix}=\mathbf{v}$ then $\begin{pmatrix}
	x\\y\\z
\end{pmatrix}=\mathbf{A}^{-1}\mathbf{v}$
\begin{description}
	\item[Consistent system of linear equations:] there is at least one set of values that satisfies all the equations simultaneously
	\item[Inconsistent:] such set of values does not exist
\end{description}

\subsection{Possible outcomes of solutions}
\includegraphics[width=0.75\textwidth]{equations}\\
\includegraphics[width=0.5\textwidth]{equations2}

\chapter{Trigonometry and modelling}
\section{Trigonometry formulae}
\subsection{Addition / subtraction}
\begin{itemize}
    \item $\sin (A \pm B) = \sin A \cos B \pm \cos A \sin B$
    \item $\cos (A \pm B) = \cos A \cos B \mp \sin A \sin B$
    \item $\tan (A \pm B) = \dfrac{\tan A \pm \tan B}{1 \mp \tan A \tan B}$
\end{itemize}
\subsection{Sum to product identities}
\begin{itemize}
    \item $\sin A + \sin B = 2\sin \dfrac{A+B}{2} \cos\dfrac{A-B}{2}$
    \item $\sin A - \sin B = 2\cos \dfrac{A+B}{2} \sin\dfrac{A-B}{2}$
    \item $\cos A + \cos B = 2\cos \dfrac{A+B}{2} \cos\dfrac{A-B}{2}$
    \item $\cos A - \cos B = -2\sin \dfrac{A+B}{2} \sin\dfrac{A-B}{2}$
\end{itemize}
\subsection{Double angle}
\begin{itemize}
    \item $\sin 2A = 2\sin A \cos A$
    \item $\cos 2A = \cos^2 A - \sin^2 A$
    \item $\tan 2A = \dfrac{2\tan A}{1-\tan^2 A}$
\end{itemize}
\subsection{Power descending}
(Derive from double angle)
\begin{itemize}
    \item $\sin A \cos A = \dfrac{\sin2A}{2}$
    \item $\sin^2 A = \dfrac{1-\cos2A}{2}$
    \item $\cos^2 A = \dfrac{1+\cos2A}{2}$
\end{itemize}
\subsection{Half angle}
\begin{itemize}
    \item $\sin \dfrac{A}{2} = \pm \sqrt{\dfrac{1-\cos A}{2}}$
    \item $\cos \dfrac{A}{2} = \pm \sqrt{\dfrac{1+\cos A}{2}}$
    \item $\tan \dfrac{A}{2} = \pm \sqrt{\dfrac{1-\cos A}{1+\cos A}} = \dfrac{1-\cos A}{\sin A} = \dfrac{\sin A}{1+\cos A}$
\end{itemize}

\section{Trigonometric equations}
\subsection{Principal values}
The angle that you get when you use the inverse trigonometric functions on the calculator
\begin{itemize}
    \item $\sin^{-1}$: $-\dfrac{\pi}{2}\leq\theta\leq\dfrac{\pi}{2}$
    \item $\cos^{-1}$: $0\leq\theta\leq\pi$
    \item $\tan^{-1}$: $-\dfrac{\pi}{2}\leq\theta\leq\dfrac{\pi}{2}$
\end{itemize}

\chapter{Parametric equations}
\section{Hyperbolic function definitions}
\subsection{$\sinh x$}
\begin{description}
	\item[Definition] $\sinh x = \dfrac{e^x-e^{-x}}{2}$
	\item[Domain] $x \in \Rset$
	\item[Range] $\sinh x \in \Rset$
	\item[Asymptotes] $x\rightarrow +\infty$, $y\rightarrow\dfrac{e^x}{2}$; $x\rightarrow -\infty$, $y\rightarrow -\dfrac{e^{-x}}{2}$
	\item[x-intercept] $(0,0)$
	\item[y-intercept] $(0,0)$	
\end{description}

\subsection{$\cosh x$}
\begin{description}
	\item[Definition] $\cosh x = \dfrac{e^x+e^{-x}}{2}$
	\item[Domain] $x \in \Rset$
	\item[Range] $\cosh x \geq 1$
	\item[Asymptotes] $x\rightarrow +\infty$, $y\rightarrow\dfrac{e^x}{2}$; $x\rightarrow -\infty$, $y\rightarrow\dfrac{e^{-x}}{2}$
	\item[x-intercept] No
	\item[y-intercept] $(0,1)$
\end{description}

\subsection{$\tanh x$}
\begin{description}
	\item[Definition] $\tanh x = \dfrac{\sinh x}{\cosh x}=\dfrac{e^x-e^{-x}}{e^x+e^{-x}}$
	\item[Domain] $x \in \textbf{R}$
	\item[Range] $-1 < \tanh x < 1$
	\item[Asymptotes] $x\rightarrow +\infty$, $y\rightarrow 1$; $x\rightarrow -\infty$, $y\rightarrow -1$
	\item[x-intercept] $(0,0)$
	\item[y-intercept] $(0,0)$
\end{description}

\subsection{Function graphs}
\includegraphics[width=\linewidth]{images/hyperbolic_graphs}

\section{Inverse hyperbolic functions}
\begin{itemize}
	\item $\arsinh x = \ln \left[x+\sqrt{x^2+1}\right]$
	\item $\arcosh x = \ln \left[x+\sqrt{x^2-1}\right] \:\: (x \geq 1)$
	\item $\artanh x = \ln \left[\dfrac{1+x}{1-x}\right] \:\: (-1 < x < 1)$
\end{itemize}

\subsection{Proof}
\begin{example}
	Show that $\arsinh x = \ln \left[x+\sqrt{x^2+1}\right]$
\end{example}

\begin{solution}
	Let $y=\arsinh x$
	\begin{align*}
		\sinh y &= x\\
		\dfrac{e^y-e^{-y}}{2} &= x\\
		e^y-e{-y} &= 2x\\
		e^{2y} - 1 &= 2x e^y\\
		(e^y-x)^2 &= x^2 + 1\\
		e^y &= x \pm \sqrt{x^2 + 1}
	\end{align*}
	$e^y = x + \sqrt{x^2 + 1}$ \text{since $\sqrt{x^2 + 1} > 0$ so it makes $e^y$ negative which is impossible}\\
	Hence $y = \ln \left[x+\sqrt{x^2+1}\right]$ so $\arsinh x = \ln \left[x+\sqrt{x^2+1}\right]$
\end{solution}

\begin{remark}
	Prove these identities by finding value of $e^y$ in quadratic equations, think about domain when deciding the sign before the square root
\end{remark}

\subsection{Graphs}
\includegraphics[width=\linewidth]{images/hyperbolic_inverse_graphs}



\section{Identities of hyperbolic functions}
Similar to trigonometric identities:
\begin{itemize}
	\item $\tanh x = \dfrac{\sinh x}{\cosh x}$
	\item $\cosh^2 x - \sinh^2 x = 1$
	\item $\tanh^2 x + \sech^2 x= 1$
	\item $\coth^2 x - \csch^2 x = 1$
\end{itemize}

\subsection{Addition}
\begin{itemize}
	\item $\sinh(x+y)=\sinh x \cosh y + \sinh y \cosh x$
	\item $\cosh(x+y)=\cosh x \cosh y + \sinh x \sinh y$
	\item $\tanh(x+y) = \dfrac{\sinh(x+y)}{\cosh(x+y)} = \dfrac{\sinh x \cosh y + \sinh y \cosh x}{\cosh x \cosh y + \sinh x \sinh y} = \dfrac{\frac{\sinh x}{\cosh x}+\frac{\sinh y}{\cosh y}}{1 + \frac{\sinh x \sinh y}{\cosh x \cosh y}}=\dfrac{\tanh x + \tanh y}{1 + \tanh x \tanh y}$
\end{itemize}

\subsection{Double angle}
\begin{itemize}
	\item $\sinh 2x = 2\sinh x \cosh x$
	\item $\cosh 2x = \cosh^2 x + \sinh^2 x = 2\cosh^2 x - 1 = 2\sinh^2 + 1$
	\item $\tanh 2x = \dfrac{2\tanh x}{1 + \tanh^2 x}$
\end{itemize}

\subsection{Power descending}
\begin{itemize}
	\item $\sinh^2 x = \dfrac{\cosh 2x - 1}{2}$
	\item $\cosh^2 x = \dfrac{\cosh 2x + 1}{2}$
\end{itemize}


\section{Calculus with hyperbolic functions}
\subsection{Differentiation}
\[(\sinh x)'=\cosh x\]
\[(\cosh x)'=\sinh x\]
\[(\tanh x)'=1-\tanh^2 x = \sech^2 x\]
\[(\csch x)'=-\coth x \csch x\]
\[(\sech x)'=-\sech x \tanh x\]
\[(\coth x)'=-\csch^2 x\]
\[(\arsinh x)' = \dfrac{1}{\sqrt{1+x^2}}\]
\[(\arcosh x)' = \dfrac{1}{\sqrt{x^2-1}}\]
\[(\artanh x)' = \dfrac{1}{1-x^2}\]


\subsection{Integration}
\[\int\sinh x \dx = \cosh x + c\]
\[\int\cosh x \dx = \sinh x + c\]
\[\int \tanh x \dx = \ln |\cosh x| + c\]
\[\int \coth x \dx = \ln |\sinh x| + c\]
\[\int \sech x \dx = \ln |-\sech x + \tanh x| + c = \ln \left|\tan\left(\frac{1}{2}x+\frac{1}{4}\pi \right)\right| + c\]
\[\int \csch x \dx = -\ln|\csch x + \coth x| + c\]
\[\int \sinh^2 x \dx = \int \frac{\cosh 2x - 1}{2} \dx = \frac{\sinh 2x - 2x}{4} + c\]
\[\int \cosh^2 x \dx = \int \frac{\cosh 2x + 1}{2} \dx = \frac{\sinh 2x + 2x}{4} + c\]
\[\int \tanh^2 x \dx = \int 1-\sech^2 x \dx = x - \tanh x + c\]
\[\int \coth^2 x \dx = \int 1+\csch^2 x \dx = x - \coth x + c\]
\[\int \sinh x \cosh x \dx = \int \frac{\sinh 2x}{2} \dx = \dfrac{\cosh 2x}{4} + c\]
\[\int \tanh x \sech x \dx = -\sech x + c\]
\[\int \coth x \csch x \dx = -\csch x + c\]
\[\int \frac{1}{\sqrt{x^2-a^2}} \dx= \arcosh \left(\frac{x}{a}\right) + c = \ln \left(x + \sqrt{x^2-a^2}\right) + c \:\:  (x>a)\]
\[\int \frac{1}{\sqrt{a^2+x^2}} \dx= \arsinh \left(\frac{x}{a}\right) + c = \ln \left(x + \sqrt{x^2+a^2}\right) + c \]
\[\int \frac{1}{a^2-x^2} \dx=\frac{1}{a}\artanh \left(\frac{x}{a}\right) + c = \frac{1}{2a}\ln \left|\frac{a+x}{a-x}\right| + c\]


\chapter{Differentiation}
\section{SUVAT equations}
\begin{itemize}
	\item $s=ut+\dfrac{1}{2}at^2$
	\item $s=vt-\dfrac{1}{2}at^2$
	\item $v=u+at$
	\item $v^2=u^2+2as$
	\item $s=\dfrac{1}{2}(u+v)t$
\end{itemize}


\chapter{Numerical methods}
\section{Locating roots}
\subsection{Method}
If a function $f(x)$ is continuous on the interval $[a,b]$ and $f(a)$ and $f(b)$ have opposite signs, then $f(x)$ has at least one root, $x$, which satisfies $a<x<b$
\subsection{How change of sign can fail}
\begin{itemize}
    \item When the interval is too large sign may not change as there may be an even number of roots
    \item If the function is not continuous, sign may change but there may be an asymptote e.g. reciprocal graph
\end{itemize}

\subsection{Model answer}
\begin{itemize}
    \item $f\left( a \right) = \dots$
    \item $f\left( b \right) = \dots$
    \item There is a change of sign in the interval $[a, b]$ and $f\left( x \right)$ is continuous so there is at least one root in this interval
\end{itemize}

\section{Iteration diagrams}
\subsection{Convergent diagrams}
\includegraphics[width=\linewidth]{Cobwebstaircase.png}
\subsection{Divergent diagrams}
\includegraphics[width=0.6\linewidth]{fixed_point_iteration_im3.png}

\chapter{Integration}
\section{Finding distance travelled}
\begin{itemize}
	\item Use graph to show sign changes during the interval
	\item Remember to account for periods with negative velocity
\end{itemize}

\chapter{Vectors}
\section{The first principle}
$$f'(x) = \lim\limits_{n \to 0}\dfrac{f(x+h)-f(x)}{h}$$


\part{Applied 1 Statistics}
\chapter{Data collection}
\section{Finding areas of shapes}
\begin{itemize}
    \item Area of triangle $\bigtriangleup ABC$ = $\dfrac{1}{2}|\overrightarrow{AB}\times\overrightarrow{AC}|$
    \item Area of parallelogram $ABCD$ = $|(b-a) \times (d-a)|=|(a\times b)+(b \times d) + (d \times a)|$ ($A$, $B$, $C$, $D$ have position vector $a$, $b$, $c$, $d$ respectively)
\end{itemize}
\section{Scalar triple product}
\begin{itemize}
    \item Volume of parallelepiped = $a\cdot(b\times c)$ ($a$, $b$, $c$ = 3 different sides)
    \item Volume of tetrahedron $ABCD$ = $\dfrac{1}{6}|\overrightarrow{AD}\cdot(\overrightarrow{AB}\times\overrightarrow{AC})|$
\end{itemize}
\section{Straight lines}
\subsection{Vector equation of line}
\begin{itemize}
    \item $(\vec{r}-\vec{a})\times\vec{b}=0$
    \item $\vec{a}$ = position vector of a point on line, $\vec{b}$ = directional vector
\end{itemize}
\subsection{Direction cosines}
For straight line $(r-a)\times b=0$, where $a=x\vec{i}+y\vec{j}+z\vec{k}$ and the line makes angle $\alpha$, $\beta$ and $\gamma$ with the positive $x$-, $y$- and $z$-axes respectively:
\begin{itemize}
    \item $l=\cos\alpha = \dfrac{x}{|a|}$
    \item $m=\cos\beta = \dfrac{y}{|a|}$
    \item $n=\cos\gamma = \dfrac{z}{|a|}$
    \item $l^2+m^2+n^2=1$
\end{itemize}

\chapter{Measurements of location and spread}
% \section{Circle}
% \subsection{Definition of circle}
% \subsubsection{Cartesian form}
% \begin{itemize}
%     \item $(x-a)^2+(y-b)^2=r^2$
% \end{itemize}

% \subsubsection{Parametric form}
% \begin{itemize}
%     \item $x=a+r\cos\theta$
%     \item $y=b+r\sin\theta$
% \end{itemize}

% \subsubsection{Polar form}
% \begin{itemize}
%     \item $r=a$
%     \item $r=a\sin\theta$
%     \item $r=a\cos\theta$
% \end{itemize}

\section{Parabola}
\subsection{Graph}
\includegraphics[width=0.6\textwidth]{parabola.png}
\begin{itemize}
    \item Symmetrix about the $x$-axis
    \item Focus at $(a, 0)$
    \item Vertex at $(0, 0)$
\end{itemize}
\subsection{Definition}
\begin{itemize}
    \item The locus of points that are the \textbf{same distance} from a fixed point, $S$,
          called the \textbf{focus}, and a fixed straight line called the \textbf{directrix}
    \item $\dfrac{\text{distance to foci}}{\text{distance to directrix}} = e = 1$
\end{itemize}
\subsection{Cartesian equation}
\begin{itemize}
    \item $y^2=4ax$ ($a>0$)
\end{itemize}
\subsection{Parametric equation}
\begin{itemize}
    \item $x=at^2$
    \item $y=2at$
    \item $t\in \Rset$
\end{itemize}
\subsection{Eccentricity}
\begin{itemize}
    \item $e=1$
\end{itemize}
\subsection{Directrix}
\begin{itemize}
    \item The directrix has equation $x+a=0$
\end{itemize}
\subsection{Tangents and normals}
\begin{itemize}
    \item $\dfrac{\dy}{\dx} = \frac{1}{t} = \frac{2a}{y}$
    \item Equation of tangent: $ty=x+at^2$ at $P(at^2, 2at)$
    \item Equation of normal: $y+tx=2at+at^3$ at $P(at^2, 2at)$
\end{itemize}

\section{Rectangular hyperbolas}
\subsection{Graph}
\includegraphics[width=0.3\textwidth]{rectangular_hyperbola.png}
\begin{itemize}
    \item Asymptotes at $x=0$ and $y=0$ ($x$ and $y$-axis)
\end{itemize}
\subsection{Definition}
\begin{itemize}
    \item The locus of points that are the \textbf{same distance} from a fixed point, $S$,
          called the \textbf{focus}, and a fixed straight line called the \textbf{directrix}
    \item $\dfrac{\text{distance to foci}}{\text{distance to directrix}} = e = 1$
\end{itemize}
\subsection{Cartesian equation}
\begin{itemize}
    \item $xy=c^2$ ($c>0$)
\end{itemize}
\subsection{Parametric equation}
\begin{itemize}
    \item $x=ct$
    \item $y=\frac{c}{t}$
    \item $t \neq 0, t\in\Rset$
\end{itemize}
\subsection{Eccentricity}
\begin{itemize}
    \item $e=\sqrt{2}$
\end{itemize}
\subsection{Directrix}
\begin{itemize}
    \item $x+y=\pm c\sqrt{2}$
\end{itemize}
\subsection{Tangents and normals}
\begin{itemize}
    \item Equation of tangent: $x+t^2y=2ct$ at $P(ct, \frac{c}{t})$
    \item Equation of normal: $t^3x-ty=c(t^4-1)$ at $P(ct, \frac{c}{t})$
\end{itemize}

\chapter{Representations of data}
\section{Hooke's law}
\begin{itemize}
    \item $\text{Tension produced}\propto x$ $\rightarrow$ $T=kx$, where $k$ is a constant
    \item $k$ depends on the unstretched length of the string or spring ($l$) and the \textbf{modulus of elasticity of the string or spring} ($\lambda$, unit = N)
    \item Hence $T=\frac{\lambda x}{l}$
    \item[*] Can also be applied if the string or spring is compressed
\end{itemize}

\section{Elastic energy}
\begin{itemize}
    \item Work done in stretching an elastic string or spring of modulus of elasticity $\lambda$ from its natural length $l$ to a length of $(l+x)$ = $\frac{\lambda x^2}{2l}$
    \item Elastic potential energy stored = amount of energy used to stretch the spring to a length of $(l+x)$ = $\frac{1}{2}kx^2$ = $\frac{\lambda x^2}{2l}$
    \item[*] Can also be applied when an elastic string or spring is compressed
\end{itemize}

\chapter{Correlation}
\section{Vieta's Law}
For $a_{n}x^{n}+a_{n-1}x^{n-1}+\cdots +a_{1}x+a_{0}=0$:
\begin{itemize}
    \item $\sum x_i=-\frac{a_{n-1}}{a_n}$
    \item $\sum x_ix_j=\frac{a_{n-2}}{a_n}$
    \item $\sum x_ix_jx_k = -\frac{a_{n-3}}{a_n}$
    \item $\sum _{1\leq i_{1}<i_{2}<\cdots <i_{k}\leq n}\left(\prod _{j=1}^{k}r_{i_{j}}\right)=(-1)^{k}{\frac {a_{n-k}}{a_{n}}}$
    \item $\prod x_i=(-1)^n\frac{a_0}{a_n}$
\end{itemize}

\chapter{Probability}
\section{Resolving forces}
\begin{enumerate}
    \item Resolve perpendicular and parallel to the direction of motion
    \item Clearly indicate the direction of motion
    \item Use $\sum F = ma$
\end{enumerate}

\section{Friction}
\begin{itemize}
    \item $F_r \leq \mu R$
    \item Friction is only \textbf{as large as it needs to be} to oppose the motion
    \item If the object is moving then $F_r = \mu R$
\end{itemize}



\chapter{Statistical distributions}
\section{Basic calculations}
\subsection{Addition / subtraction}
$(\mathbf {A}+\mathbf {B})_{i,j}=\mathbf {A}_{i,j}+{\mathbf {B}}_{i,j},\quad 1\leq i\leq m,\quad 1\leq j\leq n$\\
$(\mathbf {A}-\mathbf {B})_{i,j}=\mathbf {A}_{i,j}-{\mathbf {B}}_{i,j},\quad 1\leq i\leq m,\quad 1\leq j\leq n$
\subsection{Scalar multiplication}
$(c\mathbf{A})_{i,j}=c\cdot A_{i,j}$
\subsection{Matrix multiplication}
\includegraphics[width=0.35\textwidth]{MatrixMultiplication}

\subsection{Transposition}
$(\mathbf{A}^\mathrm{T})_{i,j}=A_{j,i}$

\section{Special matrices}
\begin{description}
	\item[Square matrix:] The number of rows and columns are the same
	\item[Zero matrix:] All of the elements are zero
	\item[Identity matrix:] A square matrix in which all the elements on the leading diagonal are 1 and the remaining elements are 0, denoted by $\mathbf{I}_k$ for $k\times k$ identity matrix
\end{description}


\section{Determinants}

\subsection{$2\times2$ matrices}
$\begin{vmatrix}a&b\\c&d\end{vmatrix}=ad-bc$

\subsection{$3\times3$ matrices}

$\begin{vmatrix}a&b&c\\d&e&f\\g&h&i\end{vmatrix}=a\begin{vmatrix}e&f\\h&i\end{vmatrix}-b\begin{vmatrix}d&f\\g&i\end{vmatrix}+c\begin{vmatrix}d&e\\g&h\end{vmatrix}=aei+bfg+cdh-ceg-bdi-afh$

\subsection{Singular matrices}
\begin{itemize}
	\item Singular matrices are square matrices with a determinant of 0
	\item It does not have an inverse
	\item If $\mathbf{A}$ and $\mathbf{B}$ are non-singular matrices, then $(\mathbf{AB})^{-1}=\mathbf{B}^{-1}\mathbf{A}^{-1}$
\end{itemize}



\subsection{Properties of determinants}
\begin{itemize}
	\item $\det(\mathbf{AB})=\det(\mathbf{A})\det(\mathbf{B})=\det(\mathbf{B})\det(\mathbf{A})=\det(\mathbf{BA})$
	\item $\det(k\mathbf{A})=k^n\det(\mathbf{A})$ ($\mathbf{A}$ is a $n\times n$ matrix)
\end{itemize}


\section{Inverse matrices}
\subsection{$2\times2$ matrices}
$\begin{bmatrix}
	a & b\\c & d
\end{bmatrix}^{-1}=\dfrac{1}{ad-bc}\begin{bmatrix}
	d & -b\\-c & a
\end{bmatrix}$

\subsection{$3\times3$ matrices}
$\mathbf {A} ^{-1}={\begin{bmatrix}a&b&c\\d&e&f\\g&h&i\\\end{bmatrix}}^{-1}={\frac {1}{\det(\mathbf {A} )}}{\begin{bmatrix}\,A&\,B&\,C\\\,D&\,E&\,F\\\,G&\,H&\,I\\\end{bmatrix}}^{\mathrm {T} }={\frac {1}{\det(\mathbf {A} )}}{\begin{bmatrix}\,A&\,D&\,G\\\,B&\,E&\,H\\\,C&\,F&\,I\\\end{bmatrix}}$\\
$\begin{alignedat}{6}A&={}&(ei-fh),&\quad &D&={}&-(bi-ch),&\quad &G&={}&(bf-ce),\\B&={}&-(di-fg),&\quad &E&={}&(ai-cg),&\quad &H&={}&-(af-cd),\\C&={}&(dh-eg),&\quad &F&={}&-(ah-bg),&\quad &I&={}&(ae-bd).\\\end{alignedat}$

\subsection{Solving equations with matrices}
If $\mathbf{A}\begin{pmatrix}
	x\\y\\z
\end{pmatrix}=\mathbf{v}$ then $\begin{pmatrix}
	x\\y\\z
\end{pmatrix}=\mathbf{A}^{-1}\mathbf{v}$
\begin{description}
	\item[Consistent system of linear equations:] there is at least one set of values that satisfies all the equations simultaneously
	\item[Inconsistent:] such set of values does not exist
\end{description}

\subsection{Possible outcomes of solutions}
\includegraphics[width=0.75\textwidth]{equations}\\
\includegraphics[width=0.5\textwidth]{equations2}

\chapter{Hypothesis testing}
\subsection{Polar form}
\begin{itemize}
	\item $P=(r,\theta)$
	\item $r=f(\theta)$
\end{itemize}

\subsection{Converting between different forms of coordinates}
\subsubsection{Polar to Cartesian form}
\begin{itemize}
	\item $x=r\cos\theta$
	\item $y=r\sin\theta$
\end{itemize}
\subsubsection{Cartesian to polar form}
\begin{itemize}
	\item $r=\sqrt{x^2+y^2}$
	\item $\tan\theta=\dfrac{y}{x}$
\end{itemize}


\subsection{Sketching curves for polar equations}
\begin{enumerate}
	\item Plot points
	\item Consider symmetry
	\item Convert to Cartesian form
\end{enumerate}
\subsubsection{Circle polar form}
\begin{itemize}
	\item 
\end{itemize}
\subsubsection{Polar form of lines}
\subsubsection{Cardioids}
\subsubsection{Lima con}
\subsubsection{Rose}
\subsubsection{Lemniscates}
\subsubsection{Spiral}

\subsection{Testing for symmetry}


\subsection{Finding area enclosed}
Area = $\dfrac{1}{2}\int_{\alpha}^{\beta} r^2d\theta$

\part{Applied 1 Mechanics}
\setcounter{chapter}{7}
\chapter{Modelling in mechanics}
\section{Hyperbolic function definitions}
\subsection{$\sinh x$}
\begin{description}
	\item[Definition] $\sinh x = \dfrac{e^x-e^{-x}}{2}$
	\item[Domain] $x \in \Rset$
	\item[Range] $\sinh x \in \Rset$
	\item[Asymptotes] $x\rightarrow +\infty$, $y\rightarrow\dfrac{e^x}{2}$; $x\rightarrow -\infty$, $y\rightarrow -\dfrac{e^{-x}}{2}$
	\item[x-intercept] $(0,0)$
	\item[y-intercept] $(0,0)$	
\end{description}

\subsection{$\cosh x$}
\begin{description}
	\item[Definition] $\cosh x = \dfrac{e^x+e^{-x}}{2}$
	\item[Domain] $x \in \Rset$
	\item[Range] $\cosh x \geq 1$
	\item[Asymptotes] $x\rightarrow +\infty$, $y\rightarrow\dfrac{e^x}{2}$; $x\rightarrow -\infty$, $y\rightarrow\dfrac{e^{-x}}{2}$
	\item[x-intercept] No
	\item[y-intercept] $(0,1)$
\end{description}

\subsection{$\tanh x$}
\begin{description}
	\item[Definition] $\tanh x = \dfrac{\sinh x}{\cosh x}=\dfrac{e^x-e^{-x}}{e^x+e^{-x}}$
	\item[Domain] $x \in \textbf{R}$
	\item[Range] $-1 < \tanh x < 1$
	\item[Asymptotes] $x\rightarrow +\infty$, $y\rightarrow 1$; $x\rightarrow -\infty$, $y\rightarrow -1$
	\item[x-intercept] $(0,0)$
	\item[y-intercept] $(0,0)$
\end{description}

\subsection{Function graphs}
\includegraphics[width=\linewidth]{images/hyperbolic_graphs}

\section{Inverse hyperbolic functions}
\begin{itemize}
	\item $\arsinh x = \ln \left[x+\sqrt{x^2+1}\right]$
	\item $\arcosh x = \ln \left[x+\sqrt{x^2-1}\right] \:\: (x \geq 1)$
	\item $\artanh x = \ln \left[\dfrac{1+x}{1-x}\right] \:\: (-1 < x < 1)$
\end{itemize}

\subsection{Proof}
\begin{example}
	Show that $\arsinh x = \ln \left[x+\sqrt{x^2+1}\right]$
\end{example}

\begin{solution}
	Let $y=\arsinh x$
	\begin{align*}
		\sinh y &= x\\
		\dfrac{e^y-e^{-y}}{2} &= x\\
		e^y-e{-y} &= 2x\\
		e^{2y} - 1 &= 2x e^y\\
		(e^y-x)^2 &= x^2 + 1\\
		e^y &= x \pm \sqrt{x^2 + 1}
	\end{align*}
	$e^y = x + \sqrt{x^2 + 1}$ \text{since $\sqrt{x^2 + 1} > 0$ so it makes $e^y$ negative which is impossible}\\
	Hence $y = \ln \left[x+\sqrt{x^2+1}\right]$ so $\arsinh x = \ln \left[x+\sqrt{x^2+1}\right]$
\end{solution}

\begin{remark}
	Prove these identities by finding value of $e^y$ in quadratic equations, think about domain when deciding the sign before the square root
\end{remark}

\subsection{Graphs}
\includegraphics[width=\linewidth]{images/hyperbolic_inverse_graphs}



\section{Identities of hyperbolic functions}
Similar to trigonometric identities:
\begin{itemize}
	\item $\tanh x = \dfrac{\sinh x}{\cosh x}$
	\item $\cosh^2 x - \sinh^2 x = 1$
	\item $\tanh^2 x + \sech^2 x= 1$
	\item $\coth^2 x - \csch^2 x = 1$
\end{itemize}

\subsection{Addition}
\begin{itemize}
	\item $\sinh(x+y)=\sinh x \cosh y + \sinh y \cosh x$
	\item $\cosh(x+y)=\cosh x \cosh y + \sinh x \sinh y$
	\item $\tanh(x+y) = \dfrac{\sinh(x+y)}{\cosh(x+y)} = \dfrac{\sinh x \cosh y + \sinh y \cosh x}{\cosh x \cosh y + \sinh x \sinh y} = \dfrac{\frac{\sinh x}{\cosh x}+\frac{\sinh y}{\cosh y}}{1 + \frac{\sinh x \sinh y}{\cosh x \cosh y}}=\dfrac{\tanh x + \tanh y}{1 + \tanh x \tanh y}$
\end{itemize}

\subsection{Double angle}
\begin{itemize}
	\item $\sinh 2x = 2\sinh x \cosh x$
	\item $\cosh 2x = \cosh^2 x + \sinh^2 x = 2\cosh^2 x - 1 = 2\sinh^2 + 1$
	\item $\tanh 2x = \dfrac{2\tanh x}{1 + \tanh^2 x}$
\end{itemize}

\subsection{Power descending}
\begin{itemize}
	\item $\sinh^2 x = \dfrac{\cosh 2x - 1}{2}$
	\item $\cosh^2 x = \dfrac{\cosh 2x + 1}{2}$
\end{itemize}


\section{Calculus with hyperbolic functions}
\subsection{Differentiation}
\[(\sinh x)'=\cosh x\]
\[(\cosh x)'=\sinh x\]
\[(\tanh x)'=1-\tanh^2 x = \sech^2 x\]
\[(\csch x)'=-\coth x \csch x\]
\[(\sech x)'=-\sech x \tanh x\]
\[(\coth x)'=-\csch^2 x\]
\[(\arsinh x)' = \dfrac{1}{\sqrt{1+x^2}}\]
\[(\arcosh x)' = \dfrac{1}{\sqrt{x^2-1}}\]
\[(\artanh x)' = \dfrac{1}{1-x^2}\]


\subsection{Integration}
\[\int\sinh x \dx = \cosh x + c\]
\[\int\cosh x \dx = \sinh x + c\]
\[\int \tanh x \dx = \ln |\cosh x| + c\]
\[\int \coth x \dx = \ln |\sinh x| + c\]
\[\int \sech x \dx = \ln |-\sech x + \tanh x| + c = \ln \left|\tan\left(\frac{1}{2}x+\frac{1}{4}\pi \right)\right| + c\]
\[\int \csch x \dx = -\ln|\csch x + \coth x| + c\]
\[\int \sinh^2 x \dx = \int \frac{\cosh 2x - 1}{2} \dx = \frac{\sinh 2x - 2x}{4} + c\]
\[\int \cosh^2 x \dx = \int \frac{\cosh 2x + 1}{2} \dx = \frac{\sinh 2x + 2x}{4} + c\]
\[\int \tanh^2 x \dx = \int 1-\sech^2 x \dx = x - \tanh x + c\]
\[\int \coth^2 x \dx = \int 1+\csch^2 x \dx = x - \coth x + c\]
\[\int \sinh x \cosh x \dx = \int \frac{\sinh 2x}{2} \dx = \dfrac{\cosh 2x}{4} + c\]
\[\int \tanh x \sech x \dx = -\sech x + c\]
\[\int \coth x \csch x \dx = -\csch x + c\]
\[\int \frac{1}{\sqrt{x^2-a^2}} \dx= \arcosh \left(\frac{x}{a}\right) + c = \ln \left(x + \sqrt{x^2-a^2}\right) + c \:\:  (x>a)\]
\[\int \frac{1}{\sqrt{a^2+x^2}} \dx= \arsinh \left(\frac{x}{a}\right) + c = \ln \left(x + \sqrt{x^2+a^2}\right) + c \]
\[\int \frac{1}{a^2-x^2} \dx=\frac{1}{a}\artanh \left(\frac{x}{a}\right) + c = \frac{1}{2a}\ln \left|\frac{a+x}{a-x}\right| + c\]


\chapter{Constant acceleration}
\section{SUVAT equations}
\begin{itemize}
	\item $s=ut+\dfrac{1}{2}at^2$
	\item $s=vt-\dfrac{1}{2}at^2$
	\item $v=u+at$
	\item $v^2=u^2+2as$
	\item $s=\dfrac{1}{2}(u+v)t$
\end{itemize}


\chapter{Forces and motion}
\section{Locating roots}
\subsection{Method}
If a function $f(x)$ is continuous on the interval $[a,b]$ and $f(a)$ and $f(b)$ have opposite signs, then $f(x)$ has at least one root, $x$, which satisfies $a<x<b$
\subsection{How change of sign can fail}
\begin{itemize}
    \item When the interval is too large sign may not change as there may be an even number of roots
    \item If the function is not continuous, sign may change but there may be an asymptote e.g. reciprocal graph
\end{itemize}

\subsection{Model answer}
\begin{itemize}
    \item $f\left( a \right) = \dots$
    \item $f\left( b \right) = \dots$
    \item There is a change of sign in the interval $[a, b]$ and $f\left( x \right)$ is continuous so there is at least one root in this interval
\end{itemize}

\section{Iteration diagrams}
\subsection{Convergent diagrams}
\includegraphics[width=\linewidth]{Cobwebstaircase.png}
\subsection{Divergent diagrams}
\includegraphics[width=0.6\linewidth]{fixed_point_iteration_im3.png}

\chapter{Variable acceleration}
\section{Finding distance travelled}
\begin{itemize}
	\item Use graph to show sign changes during the interval
	\item Remember to account for periods with negative velocity
\end{itemize}


\part{Applied 2 Statistics}
\chapter{Regression, correlation and hypothesis testing}
\section{Finding areas of shapes}
\begin{itemize}
    \item Area of triangle $\bigtriangleup ABC$ = $\dfrac{1}{2}|\overrightarrow{AB}\times\overrightarrow{AC}|$
    \item Area of parallelogram $ABCD$ = $|(b-a) \times (d-a)|=|(a\times b)+(b \times d) + (d \times a)|$ ($A$, $B$, $C$, $D$ have position vector $a$, $b$, $c$, $d$ respectively)
\end{itemize}
\section{Scalar triple product}
\begin{itemize}
    \item Volume of parallelepiped = $a\cdot(b\times c)$ ($a$, $b$, $c$ = 3 different sides)
    \item Volume of tetrahedron $ABCD$ = $\dfrac{1}{6}|\overrightarrow{AD}\cdot(\overrightarrow{AB}\times\overrightarrow{AC})|$
\end{itemize}
\section{Straight lines}
\subsection{Vector equation of line}
\begin{itemize}
    \item $(\vec{r}-\vec{a})\times\vec{b}=0$
    \item $\vec{a}$ = position vector of a point on line, $\vec{b}$ = directional vector
\end{itemize}
\subsection{Direction cosines}
For straight line $(r-a)\times b=0$, where $a=x\vec{i}+y\vec{j}+z\vec{k}$ and the line makes angle $\alpha$, $\beta$ and $\gamma$ with the positive $x$-, $y$- and $z$-axes respectively:
\begin{itemize}
    \item $l=\cos\alpha = \dfrac{x}{|a|}$
    \item $m=\cos\beta = \dfrac{y}{|a|}$
    \item $n=\cos\gamma = \dfrac{z}{|a|}$
    \item $l^2+m^2+n^2=1$
\end{itemize}

\chapter{Conditional probability}
% \section{Circle}
% \subsection{Definition of circle}
% \subsubsection{Cartesian form}
% \begin{itemize}
%     \item $(x-a)^2+(y-b)^2=r^2$
% \end{itemize}

% \subsubsection{Parametric form}
% \begin{itemize}
%     \item $x=a+r\cos\theta$
%     \item $y=b+r\sin\theta$
% \end{itemize}

% \subsubsection{Polar form}
% \begin{itemize}
%     \item $r=a$
%     \item $r=a\sin\theta$
%     \item $r=a\cos\theta$
% \end{itemize}

\section{Parabola}
\subsection{Graph}
\includegraphics[width=0.6\textwidth]{parabola.png}
\begin{itemize}
    \item Symmetrix about the $x$-axis
    \item Focus at $(a, 0)$
    \item Vertex at $(0, 0)$
\end{itemize}
\subsection{Definition}
\begin{itemize}
    \item The locus of points that are the \textbf{same distance} from a fixed point, $S$,
          called the \textbf{focus}, and a fixed straight line called the \textbf{directrix}
    \item $\dfrac{\text{distance to foci}}{\text{distance to directrix}} = e = 1$
\end{itemize}
\subsection{Cartesian equation}
\begin{itemize}
    \item $y^2=4ax$ ($a>0$)
\end{itemize}
\subsection{Parametric equation}
\begin{itemize}
    \item $x=at^2$
    \item $y=2at$
    \item $t\in \Rset$
\end{itemize}
\subsection{Eccentricity}
\begin{itemize}
    \item $e=1$
\end{itemize}
\subsection{Directrix}
\begin{itemize}
    \item The directrix has equation $x+a=0$
\end{itemize}
\subsection{Tangents and normals}
\begin{itemize}
    \item $\dfrac{\dy}{\dx} = \frac{1}{t} = \frac{2a}{y}$
    \item Equation of tangent: $ty=x+at^2$ at $P(at^2, 2at)$
    \item Equation of normal: $y+tx=2at+at^3$ at $P(at^2, 2at)$
\end{itemize}

\section{Rectangular hyperbolas}
\subsection{Graph}
\includegraphics[width=0.3\textwidth]{rectangular_hyperbola.png}
\begin{itemize}
    \item Asymptotes at $x=0$ and $y=0$ ($x$ and $y$-axis)
\end{itemize}
\subsection{Definition}
\begin{itemize}
    \item The locus of points that are the \textbf{same distance} from a fixed point, $S$,
          called the \textbf{focus}, and a fixed straight line called the \textbf{directrix}
    \item $\dfrac{\text{distance to foci}}{\text{distance to directrix}} = e = 1$
\end{itemize}
\subsection{Cartesian equation}
\begin{itemize}
    \item $xy=c^2$ ($c>0$)
\end{itemize}
\subsection{Parametric equation}
\begin{itemize}
    \item $x=ct$
    \item $y=\frac{c}{t}$
    \item $t \neq 0, t\in\Rset$
\end{itemize}
\subsection{Eccentricity}
\begin{itemize}
    \item $e=\sqrt{2}$
\end{itemize}
\subsection{Directrix}
\begin{itemize}
    \item $x+y=\pm c\sqrt{2}$
\end{itemize}
\subsection{Tangents and normals}
\begin{itemize}
    \item Equation of tangent: $x+t^2y=2ct$ at $P(ct, \frac{c}{t})$
    \item Equation of normal: $t^3x-ty=c(t^4-1)$ at $P(ct, \frac{c}{t})$
\end{itemize}

\chapter{The normal distribution}
\section{Hooke's law}
\begin{itemize}
    \item $\text{Tension produced}\propto x$ $\rightarrow$ $T=kx$, where $k$ is a constant
    \item $k$ depends on the unstretched length of the string or spring ($l$) and the \textbf{modulus of elasticity of the string or spring} ($\lambda$, unit = N)
    \item Hence $T=\frac{\lambda x}{l}$
    \item[*] Can also be applied if the string or spring is compressed
\end{itemize}

\section{Elastic energy}
\begin{itemize}
    \item Work done in stretching an elastic string or spring of modulus of elasticity $\lambda$ from its natural length $l$ to a length of $(l+x)$ = $\frac{\lambda x^2}{2l}$
    \item Elastic potential energy stored = amount of energy used to stretch the spring to a length of $(l+x)$ = $\frac{1}{2}kx^2$ = $\frac{\lambda x^2}{2l}$
    \item[*] Can also be applied when an elastic string or spring is compressed
\end{itemize}

\part{Applied 2 Mechanics}
\setcounter{chapter}{3}
\chapter{Moments}
\section{Vieta's Law}
For $a_{n}x^{n}+a_{n-1}x^{n-1}+\cdots +a_{1}x+a_{0}=0$:
\begin{itemize}
    \item $\sum x_i=-\frac{a_{n-1}}{a_n}$
    \item $\sum x_ix_j=\frac{a_{n-2}}{a_n}$
    \item $\sum x_ix_jx_k = -\frac{a_{n-3}}{a_n}$
    \item $\sum _{1\leq i_{1}<i_{2}<\cdots <i_{k}\leq n}\left(\prod _{j=1}^{k}r_{i_{j}}\right)=(-1)^{k}{\frac {a_{n-k}}{a_{n}}}$
    \item $\prod x_i=(-1)^n\frac{a_0}{a_n}$
\end{itemize}

\chapter{Forces and friction}
\section{Resolving forces}
\begin{enumerate}
    \item Resolve perpendicular and parallel to the direction of motion
    \item Clearly indicate the direction of motion
    \item Use $\sum F = ma$
\end{enumerate}

\section{Friction}
\begin{itemize}
    \item $F_r \leq \mu R$
    \item Friction is only \textbf{as large as it needs to be} to oppose the motion
    \item If the object is moving then $F_r = \mu R$
\end{itemize}



\chapter{Projectiles}
\section{Basic calculations}
\subsection{Addition / subtraction}
$(\mathbf {A}+\mathbf {B})_{i,j}=\mathbf {A}_{i,j}+{\mathbf {B}}_{i,j},\quad 1\leq i\leq m,\quad 1\leq j\leq n$\\
$(\mathbf {A}-\mathbf {B})_{i,j}=\mathbf {A}_{i,j}-{\mathbf {B}}_{i,j},\quad 1\leq i\leq m,\quad 1\leq j\leq n$
\subsection{Scalar multiplication}
$(c\mathbf{A})_{i,j}=c\cdot A_{i,j}$
\subsection{Matrix multiplication}
\includegraphics[width=0.35\textwidth]{MatrixMultiplication}

\subsection{Transposition}
$(\mathbf{A}^\mathrm{T})_{i,j}=A_{j,i}$

\section{Special matrices}
\begin{description}
	\item[Square matrix:] The number of rows and columns are the same
	\item[Zero matrix:] All of the elements are zero
	\item[Identity matrix:] A square matrix in which all the elements on the leading diagonal are 1 and the remaining elements are 0, denoted by $\mathbf{I}_k$ for $k\times k$ identity matrix
\end{description}


\section{Determinants}

\subsection{$2\times2$ matrices}
$\begin{vmatrix}a&b\\c&d\end{vmatrix}=ad-bc$

\subsection{$3\times3$ matrices}

$\begin{vmatrix}a&b&c\\d&e&f\\g&h&i\end{vmatrix}=a\begin{vmatrix}e&f\\h&i\end{vmatrix}-b\begin{vmatrix}d&f\\g&i\end{vmatrix}+c\begin{vmatrix}d&e\\g&h\end{vmatrix}=aei+bfg+cdh-ceg-bdi-afh$

\subsection{Singular matrices}
\begin{itemize}
	\item Singular matrices are square matrices with a determinant of 0
	\item It does not have an inverse
	\item If $\mathbf{A}$ and $\mathbf{B}$ are non-singular matrices, then $(\mathbf{AB})^{-1}=\mathbf{B}^{-1}\mathbf{A}^{-1}$
\end{itemize}



\subsection{Properties of determinants}
\begin{itemize}
	\item $\det(\mathbf{AB})=\det(\mathbf{A})\det(\mathbf{B})=\det(\mathbf{B})\det(\mathbf{A})=\det(\mathbf{BA})$
	\item $\det(k\mathbf{A})=k^n\det(\mathbf{A})$ ($\mathbf{A}$ is a $n\times n$ matrix)
\end{itemize}


\section{Inverse matrices}
\subsection{$2\times2$ matrices}
$\begin{bmatrix}
	a & b\\c & d
\end{bmatrix}^{-1}=\dfrac{1}{ad-bc}\begin{bmatrix}
	d & -b\\-c & a
\end{bmatrix}$

\subsection{$3\times3$ matrices}
$\mathbf {A} ^{-1}={\begin{bmatrix}a&b&c\\d&e&f\\g&h&i\\\end{bmatrix}}^{-1}={\frac {1}{\det(\mathbf {A} )}}{\begin{bmatrix}\,A&\,B&\,C\\\,D&\,E&\,F\\\,G&\,H&\,I\\\end{bmatrix}}^{\mathrm {T} }={\frac {1}{\det(\mathbf {A} )}}{\begin{bmatrix}\,A&\,D&\,G\\\,B&\,E&\,H\\\,C&\,F&\,I\\\end{bmatrix}}$\\
$\begin{alignedat}{6}A&={}&(ei-fh),&\quad &D&={}&-(bi-ch),&\quad &G&={}&(bf-ce),\\B&={}&-(di-fg),&\quad &E&={}&(ai-cg),&\quad &H&={}&-(af-cd),\\C&={}&(dh-eg),&\quad &F&={}&-(ah-bg),&\quad &I&={}&(ae-bd).\\\end{alignedat}$

\subsection{Solving equations with matrices}
If $\mathbf{A}\begin{pmatrix}
	x\\y\\z
\end{pmatrix}=\mathbf{v}$ then $\begin{pmatrix}
	x\\y\\z
\end{pmatrix}=\mathbf{A}^{-1}\mathbf{v}$
\begin{description}
	\item[Consistent system of linear equations:] there is at least one set of values that satisfies all the equations simultaneously
	\item[Inconsistent:] such set of values does not exist
\end{description}

\subsection{Possible outcomes of solutions}
\includegraphics[width=0.75\textwidth]{equations}\\
\includegraphics[width=0.5\textwidth]{equations2}

\chapter{Applications of forces}
\subsection{Polar form}
\begin{itemize}
	\item $P=(r,\theta)$
	\item $r=f(\theta)$
\end{itemize}

\subsection{Converting between different forms of coordinates}
\subsubsection{Polar to Cartesian form}
\begin{itemize}
	\item $x=r\cos\theta$
	\item $y=r\sin\theta$
\end{itemize}
\subsubsection{Cartesian to polar form}
\begin{itemize}
	\item $r=\sqrt{x^2+y^2}$
	\item $\tan\theta=\dfrac{y}{x}$
\end{itemize}


\subsection{Sketching curves for polar equations}
\begin{enumerate}
	\item Plot points
	\item Consider symmetry
	\item Convert to Cartesian form
\end{enumerate}
\subsubsection{Circle polar form}
\begin{itemize}
	\item 
\end{itemize}
\subsubsection{Polar form of lines}
\subsubsection{Cardioids}
\subsubsection{Lima con}
\subsubsection{Rose}
\subsubsection{Lemniscates}
\subsubsection{Spiral}

\subsection{Testing for symmetry}


\subsection{Finding area enclosed}
Area = $\dfrac{1}{2}\int_{\alpha}^{\beta} r^2d\theta$

\chapter{Further kinematics}
\section{Hyperbolic function definitions}
\subsection{$\sinh x$}
\begin{description}
	\item[Definition] $\sinh x = \dfrac{e^x-e^{-x}}{2}$
	\item[Domain] $x \in \Rset$
	\item[Range] $\sinh x \in \Rset$
	\item[Asymptotes] $x\rightarrow +\infty$, $y\rightarrow\dfrac{e^x}{2}$; $x\rightarrow -\infty$, $y\rightarrow -\dfrac{e^{-x}}{2}$
	\item[x-intercept] $(0,0)$
	\item[y-intercept] $(0,0)$	
\end{description}

\subsection{$\cosh x$}
\begin{description}
	\item[Definition] $\cosh x = \dfrac{e^x+e^{-x}}{2}$
	\item[Domain] $x \in \Rset$
	\item[Range] $\cosh x \geq 1$
	\item[Asymptotes] $x\rightarrow +\infty$, $y\rightarrow\dfrac{e^x}{2}$; $x\rightarrow -\infty$, $y\rightarrow\dfrac{e^{-x}}{2}$
	\item[x-intercept] No
	\item[y-intercept] $(0,1)$
\end{description}

\subsection{$\tanh x$}
\begin{description}
	\item[Definition] $\tanh x = \dfrac{\sinh x}{\cosh x}=\dfrac{e^x-e^{-x}}{e^x+e^{-x}}$
	\item[Domain] $x \in \textbf{R}$
	\item[Range] $-1 < \tanh x < 1$
	\item[Asymptotes] $x\rightarrow +\infty$, $y\rightarrow 1$; $x\rightarrow -\infty$, $y\rightarrow -1$
	\item[x-intercept] $(0,0)$
	\item[y-intercept] $(0,0)$
\end{description}

\subsection{Function graphs}
\includegraphics[width=\linewidth]{images/hyperbolic_graphs}

\section{Inverse hyperbolic functions}
\begin{itemize}
	\item $\arsinh x = \ln \left[x+\sqrt{x^2+1}\right]$
	\item $\arcosh x = \ln \left[x+\sqrt{x^2-1}\right] \:\: (x \geq 1)$
	\item $\artanh x = \ln \left[\dfrac{1+x}{1-x}\right] \:\: (-1 < x < 1)$
\end{itemize}

\subsection{Proof}
\begin{example}
	Show that $\arsinh x = \ln \left[x+\sqrt{x^2+1}\right]$
\end{example}

\begin{solution}
	Let $y=\arsinh x$
	\begin{align*}
		\sinh y &= x\\
		\dfrac{e^y-e^{-y}}{2} &= x\\
		e^y-e{-y} &= 2x\\
		e^{2y} - 1 &= 2x e^y\\
		(e^y-x)^2 &= x^2 + 1\\
		e^y &= x \pm \sqrt{x^2 + 1}
	\end{align*}
	$e^y = x + \sqrt{x^2 + 1}$ \text{since $\sqrt{x^2 + 1} > 0$ so it makes $e^y$ negative which is impossible}\\
	Hence $y = \ln \left[x+\sqrt{x^2+1}\right]$ so $\arsinh x = \ln \left[x+\sqrt{x^2+1}\right]$
\end{solution}

\begin{remark}
	Prove these identities by finding value of $e^y$ in quadratic equations, think about domain when deciding the sign before the square root
\end{remark}

\subsection{Graphs}
\includegraphics[width=\linewidth]{images/hyperbolic_inverse_graphs}



\section{Identities of hyperbolic functions}
Similar to trigonometric identities:
\begin{itemize}
	\item $\tanh x = \dfrac{\sinh x}{\cosh x}$
	\item $\cosh^2 x - \sinh^2 x = 1$
	\item $\tanh^2 x + \sech^2 x= 1$
	\item $\coth^2 x - \csch^2 x = 1$
\end{itemize}

\subsection{Addition}
\begin{itemize}
	\item $\sinh(x+y)=\sinh x \cosh y + \sinh y \cosh x$
	\item $\cosh(x+y)=\cosh x \cosh y + \sinh x \sinh y$
	\item $\tanh(x+y) = \dfrac{\sinh(x+y)}{\cosh(x+y)} = \dfrac{\sinh x \cosh y + \sinh y \cosh x}{\cosh x \cosh y + \sinh x \sinh y} = \dfrac{\frac{\sinh x}{\cosh x}+\frac{\sinh y}{\cosh y}}{1 + \frac{\sinh x \sinh y}{\cosh x \cosh y}}=\dfrac{\tanh x + \tanh y}{1 + \tanh x \tanh y}$
\end{itemize}

\subsection{Double angle}
\begin{itemize}
	\item $\sinh 2x = 2\sinh x \cosh x$
	\item $\cosh 2x = \cosh^2 x + \sinh^2 x = 2\cosh^2 x - 1 = 2\sinh^2 + 1$
	\item $\tanh 2x = \dfrac{2\tanh x}{1 + \tanh^2 x}$
\end{itemize}

\subsection{Power descending}
\begin{itemize}
	\item $\sinh^2 x = \dfrac{\cosh 2x - 1}{2}$
	\item $\cosh^2 x = \dfrac{\cosh 2x + 1}{2}$
\end{itemize}


\section{Calculus with hyperbolic functions}
\subsection{Differentiation}
\[(\sinh x)'=\cosh x\]
\[(\cosh x)'=\sinh x\]
\[(\tanh x)'=1-\tanh^2 x = \sech^2 x\]
\[(\csch x)'=-\coth x \csch x\]
\[(\sech x)'=-\sech x \tanh x\]
\[(\coth x)'=-\csch^2 x\]
\[(\arsinh x)' = \dfrac{1}{\sqrt{1+x^2}}\]
\[(\arcosh x)' = \dfrac{1}{\sqrt{x^2-1}}\]
\[(\artanh x)' = \dfrac{1}{1-x^2}\]


\subsection{Integration}
\[\int\sinh x \dx = \cosh x + c\]
\[\int\cosh x \dx = \sinh x + c\]
\[\int \tanh x \dx = \ln |\cosh x| + c\]
\[\int \coth x \dx = \ln |\sinh x| + c\]
\[\int \sech x \dx = \ln |-\sech x + \tanh x| + c = \ln \left|\tan\left(\frac{1}{2}x+\frac{1}{4}\pi \right)\right| + c\]
\[\int \csch x \dx = -\ln|\csch x + \coth x| + c\]
\[\int \sinh^2 x \dx = \int \frac{\cosh 2x - 1}{2} \dx = \frac{\sinh 2x - 2x}{4} + c\]
\[\int \cosh^2 x \dx = \int \frac{\cosh 2x + 1}{2} \dx = \frac{\sinh 2x + 2x}{4} + c\]
\[\int \tanh^2 x \dx = \int 1-\sech^2 x \dx = x - \tanh x + c\]
\[\int \coth^2 x \dx = \int 1+\csch^2 x \dx = x - \coth x + c\]
\[\int \sinh x \cosh x \dx = \int \frac{\sinh 2x}{2} \dx = \dfrac{\cosh 2x}{4} + c\]
\[\int \tanh x \sech x \dx = -\sech x + c\]
\[\int \coth x \csch x \dx = -\csch x + c\]
\[\int \frac{1}{\sqrt{x^2-a^2}} \dx= \arcosh \left(\frac{x}{a}\right) + c = \ln \left(x + \sqrt{x^2-a^2}\right) + c \:\:  (x>a)\]
\[\int \frac{1}{\sqrt{a^2+x^2}} \dx= \arsinh \left(\frac{x}{a}\right) + c = \ln \left(x + \sqrt{x^2+a^2}\right) + c \]
\[\int \frac{1}{a^2-x^2} \dx=\frac{1}{a}\artanh \left(\frac{x}{a}\right) + c = \frac{1}{2a}\ln \left|\frac{a+x}{a-x}\right| + c\]


\end{document}