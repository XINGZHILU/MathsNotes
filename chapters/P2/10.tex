\section{Locating roots}
\subsection{Method}
If a function $f(x)$ is continuous on the interval $[a,b]$ and $f(a)$ and $f(b)$ have opposite signs, then $f(x)$ has at least one root, $x$, which satisfies $a<x<b$
\subsection{How change of sign can fail}
\begin{itemize}
    \item When the interval is too large sign may not change as there may be an even number of roots
    \item If the function is not continuous, sign may change but there may be an asymptote e.g. reciprocal
\end{itemize}

\section{Iteration}
\begin{itemize}
    \item Iterative formula can be found by rewriting $f(x)=0$ into $x=g(x)$, then $x_{n+1}=g(x_n)$
    \item Find the value of $x_1$, $x_2$, $x_3$, etc.
    \item If converge (get closer to the root from the same direction / alternate above and below the root) then solution can be found
    \item Test for roots using the change of sign of function $h(x)=g(x)-x$
\end{itemize}

\section{The Newton-Raphson method}
\begin{itemize}
    \item $x_{n+1}=x_n-\dfrac{f(x_n)}{f'(x_n)}$
    \item Find approximation to $x$ decimal places
    \item Show accurate to $x$ decimal places: use change in sign to show
    \item[$\star$] If any value of $x_i$ is at a turning point then the method will fail as $f'(x)=0$ which results in division by zero
\end{itemize}

\section{The trapezium rule}
$\int_{a}^{b}y\:dx \approx \dfrac{1}{2} h (y_0+2(y_1+y_2+\dots+y_{n-1})+y_n)$ where $h=\dfrac{b-a}{n}$ and $y_i=f(a+ih)$
