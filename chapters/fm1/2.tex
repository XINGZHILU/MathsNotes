\section{Work and energy}
\begin{description}
    \item[Work done:] $w=Fd=\text{force}\times\text{distance moved in the direction of the force}=\text{change in kinetic energy}$
    \item[Kinetic energy:] $\text{K.E.}=\dfrac{1}{2}mv^2$
    \item[Potential energy:] $\text{P.E.}=mgh$
    \item[*] You must choose a \textbf{zero} level of potential energy before calculating a particle's potential energy
\end{description}



\section{Conservation of mechanical energy}
When no external forces (other than gravity) do work on a particle during its motion, the sum of the particle's \textbf{kinetic energy and (gravitational and elastic) potential energy} remains constant

\section{Work-energy principle}
The change in the total energy of a particle is equal to the work done on the particle

\section{Power}
\begin{description}
    \item[Definition:] Power is the rate of doing work
    \item[Equation:] $\text{Power}=Fv=\text{driving force produced by the engine}\times\text{velocity}$
\end{description}