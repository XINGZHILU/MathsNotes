\section{Definitions}
\begin{description}
    \item[Hypothesis] A statement about the value of a population parameter
    \item[Test statistic] A value computer from sample data
    \item[Null hypothesis ($H_0$)] The hypothesis assumed to be correct ($\theta=\theta_0$)
    \item[Alternative hypothesis ($H_1$)] Tells you about the parameter if $H_0$ is rejected as a result of the test ($\theta \neq \theta_0$ / $\theta>\theta_0$ (right tail) / $\theta<\theta_0$ (left tail))
    \item[Significance level ($\alpha$)] Probability of rejecting $H_0$ when assuming $H_0$ is true
    \item[Critical region] A region of the probability distribution which, if the test statistic falls within it, would cause you to reject the null hypothesis
    \item[Critical value] The first value to fall inside the critical region / a value that is compared to the test statistic to determine whether to reject $H_0$
    \item[Acceptance region] The rejection region for $H_1$ in the testing of a hypothesis
    \item[Actual significance level] The probability of incorrectly rejecting the null hypothesis (when $H_0$ is actually true)
\end{description}

\section{Test on proportion / probability of success assuming binomial distribution}
$t^*$ = test statistics
\subsection{By critical value}
\begin{itemize}
    \item One tailed: if stats test $t^* > \text{cv}$ or $t^* < \text{cv}$ (depends on right / left tail): reject $H_0$, else accept $H_0$
    \item Two tailed: if stats test $t^* > \text{upper cv}$ or $t^* < \text{lower cv}$: reject $H_0$, else accept $H_0$ (For 2 tailed tests the probability used for calculating cv at the end of each tail = $\dfrac{\alpha}{2}$)
\end{itemize}


\subsection{By $p$ value}
\begin{itemize}
    \item One tailed: if $P(t\geq t^*) < \alpha$: reject $H_0$, else accept $H_0$
    \item Two tailed: if $P(t\geq t^*) < \dfrac{\alpha}{2}$ or $P(t\leq t^*) > \dfrac{\alpha}{2}$: reject $H_0$, else accept $H_0$
\end{itemize}

\begin{comment}
\subsection{Presenting the solution}
\begin{enumerate}
    \item Define the test statistic, $X$ and the parameter, $p$
    \item Formulate a model for $X$ ($X \sim B(n, p)$)
    \item Write down $H_0$ and $H_1$ as statements involving $p$ (these determine whether your test is one or two-tailed)
    \item Specify the significance level
    \item Decide whether to reject $H_0$ by comparing $t^*$ to critical value or finding the probability of $t^*$ taking $x$ or a more extreme value
    \item State your conclusion: there is / is not sufficient evidence to reject $H_0$ + a statement \textbf{in the context of the question}
\end{enumerate}

\end{comment}


\section{Two tailed tests}
\begin{itemize}
    \item Halve the significance value to find out the critical region at each end unless otherwise specified
    \item Notice if the question asks for the probability in each tail to be \textbf{as close to $\dfrac{\alpha}{2}$ as possible}
    \item Always use 2 tailed tests if whether testing for increase / decrease in $p$ is not specified
\end{itemize}

\section{Example responses}
\subsection{One tailed + critical region}
\begin{example}
    A single observation is taken from $X\sim B(10, p)$ and $x=1$ is obtained. Use this value to test $H_0: p=0.4$ against $H_1:p<0.4$ using a $5\%$ significance level
\end{example}

\begin{solution}
    $H_0: p=0.4$\\
    $H_1:p<0.4$\\
    Test statistic: $x=1$\\
    $\text{Significance level} = 5\%$\\
    One-tailed test\\
    $P(X \leq c_1) < 0.05$\\
    $P(X \leq 1) = 0.0463$ ($P(X \leq 2) = 0.1672$ too big)\\
    $c_1 = 1$ so critical region is $X\leq 1$\\
    $x=1$ lies in the critical region, so evidence suggests rejecting $H_0$ at $5\%$ significance level
\end{solution}

\subsection{One tailed + $p$ value}
\begin{example}
    A single observation is taken from $X\sim B(10, p)$ and $x=5$ is obtained. Use this value to test $H_0: p=0.25$ against $H_1:p>0.25$ using a $5\%$ significance level
\end{example}

\begin{solution}
    $H_0: p=0.25$\\
    $H_1:p>0.25$\\
    Test statistic: $x=5$\\
    $\text{Significance level} = 5\%$\\
    One-tailed test
    \begin{align*}
        P(X\geq 5) & = 1-P(X\leq 4) \\ &= 0.0781
    \end{align*}
    Compare $p$-value with significance level: $0.0781 > 0.05$\\
    It is not significant so no evidence to reject $H_0$ at the 5\% significance level
\end{solution}

\subsection{Two tailed - find critical region when using `probability as close to'}
\begin{example}
    $Y \sim B(25, p)$, given that $H_0: p=0.42$, $H_1:p \neq 0.42$, find the critical region for the test using $10\%$ significance level, \textbf{the probability in each tail should be as close to $5\%$ as possible}
\end{example}

\begin{solution}
    $P(Y\leq c_1)$ as close to 0.05 as possible\\
    $P(Y\leq 6) = 0.0495 < 0.05$ - closest to $0.05$ so $c_1 = 6$\\
    $P(Y\leq 7) = 0.1106 > 0.05$\\\\

    $P(Y\geq c_2)$ as close to 0.05 as possible $\rightarrow$ $1-P(Y\leq c_2 -1)$ as close to 0.05 as possible $\rightarrow$ $P(Y\leq c_2 -1)$ as close to 0.95 as possible\\
    $P(Y\leq 14) = 0.9465 < 0.95$ - closest to $0.05$ so $c_2 = 14+1 = 15$\\
    $P(Y\leq 15) = 0.19779 > 0.95$
\end{solution}
