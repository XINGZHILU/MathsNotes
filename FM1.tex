\documentclass[oneside, fleqn, 11pt]{book}
\usepackage[a4paper, total={7.2in, 10.5in}]{geometry}
\usepackage{tikz}
\usetikzlibrary{calc}
\usepackage{setspace}
\usepackage{graphicx}
\usepackage{amsmath}
\usepackage{pgfplots}
\graphicspath{ {./images/} }
\usepackage{bookmark}
\setcounter{tocdepth}{0}

\DeclareMathOperator\dx{\mathrm{d}\mathit{x}}
\DeclareMathOperator\dy{\mathrm{d}\mathit{y}}
\DeclareMathOperator\dt{\mathrm{d}\mathit{t}}
\DeclareMathOperator\dv{\mathrm{d}\mathit{v}}
\DeclareMathOperator\dtheta{\mathrm{d}\mathit{\theta}}
\DeclareMathOperator\cis{cis}
\DeclareMathOperator\sech{sech}
\DeclareMathOperator\csch{csch}
\DeclareMathOperator\arsinh{arsinh}
\DeclareMathOperator\arcosh{arcosh}
\DeclareMathOperator\artanh{artanh}
\DeclareMathOperator\Nset{\mathbb{N}}
\DeclareMathOperator\Zset{\mathbb{Z}}
\DeclareMathOperator\Qset{\mathbb{Q}}
\DeclareMathOperator\Rset{\mathbb{R}}
\DeclareMathOperator\Iset{\mathbb{I}}
\DeclareMathOperator\Cset{\mathbb{C}}

%\usepackage[T1]{fontenc}
\usepackage{mathptmx}

\hypersetup{
	colorlinks   = true, %Colours links instead of ugly boxes
	urlcolor     = blue, %Colour for external hyperlinks
	linkcolor    = black, %Colour of internal links
	citecolor   = red %Colour of citations
}

\usepackage{hyperref}
\usepackage{blindtext}
\counterwithin*{chapter}{part}
\newcommand*{\Part}[2][\partheading]{%
  \refstepcounter{part}%
  \def\partheading{#2}%
  \part*{#2}%
  \addcontentsline{toc}{part}{#1}%
}

\newcommand{\tikzAngleOfLine}{\tikz@AngleOfLine}
\def\tikz@AngleOfLine(#1)(#2)#3{%
	\pgfmathanglebetweenpoints{%
		\pgfpointanchor{#1}{center}}{%
		\pgfpointanchor{#2}{center}}
	\pgfmathsetmacro{#3}{\pgfmathresult}%
}

\title{A Level Further Mathematics Notes - FM1}
\author{Xingzhi Lu}
\date{}

\begin{document}
\maketitle
\everymath{\displaystyle}
\tableofcontents

\chapter{Momentum and impulse}
\section{Equations}
\begin{description}
    \item[Momentum:] $p=mv$ (unit = $\text{kg m s}^{-1}$)
    \item[Impulse:] $I=\Delta mv=mv-mu=Ft$ (unit = $\text{N s}$)
\end{description}

\section{Conservation of momentum}
\begin{itemize}
    \item Momentum is always conserved in any interaction where no external forces act
    \item Elastic collision: $m_1u_1+m_2u_2=m_1v_1+m_2v_2$
    \item Sticking together: $m_1u_1+m_2u_2=(m_1+m_2)v_{1+2}$
    \item Explosion: $m_1v_1+m_2v_2=0$
\end{itemize}

\section{Momentum as a vector}
Calculate each direction independently

\chapter{Work, energy and power}
% \section{Circle}
% \subsection{Definition of circle}
% \subsubsection{Cartesian form}
% \begin{itemize}
%     \item $(x-a)^2+(y-b)^2=r^2$
% \end{itemize}

% \subsubsection{Parametric form}
% \begin{itemize}
%     \item $x=a+r\cos\theta$
%     \item $y=b+r\sin\theta$
% \end{itemize}

% \subsubsection{Polar form}
% \begin{itemize}
%     \item $r=a$
%     \item $r=a\sin\theta$
%     \item $r=a\cos\theta$
% \end{itemize}

\section{Parabola}
\subsection{Graph}
\includegraphics[width=0.6\textwidth]{parabola.png}
\begin{itemize}
    \item Symmetrix about the $x$-axis
    \item Focus at $(a, 0)$
    \item Vertex at $(0, 0)$
\end{itemize}
\subsection{Definition}
\begin{itemize}
    \item The locus of points that are the \textbf{same distance} from a fixed point, $S$,
          called the \textbf{focus}, and a fixed straight line called the \textbf{directrix}
    \item $\dfrac{\text{distance to foci}}{\text{distance to directrix}} = e = 1$
\end{itemize}
\subsection{Cartesian equation}
\begin{itemize}
    \item $y^2=4ax$ ($a>0$)
\end{itemize}
\subsection{Parametric equation}
\begin{itemize}
    \item $x=at^2$
    \item $y=2at$
    \item $t\in \Rset$
\end{itemize}
\subsection{Eccentricity}
\begin{itemize}
    \item $e=1$
\end{itemize}
\subsection{Directrix}
\begin{itemize}
    \item The directrix has equation $x+a=0$
\end{itemize}
\subsection{Tangents and normals}
\begin{itemize}
    \item $\dfrac{\dy}{\dx} = \frac{1}{t} = \frac{2a}{y}$
    \item Equation of tangent: $ty=x+at^2$ at $P(at^2, 2at)$
    \item Equation of normal: $y+tx=2at+at^3$ at $P(at^2, 2at)$
\end{itemize}

\section{Rectangular hyperbolas}
\subsection{Graph}
\includegraphics[width=0.3\textwidth]{rectangular_hyperbola.png}
\begin{itemize}
    \item Asymptotes at $x=0$ and $y=0$ ($x$ and $y$-axis)
\end{itemize}
\subsection{Definition}
\begin{itemize}
    \item The locus of points that are the \textbf{same distance} from a fixed point, $S$,
          called the \textbf{focus}, and a fixed straight line called the \textbf{directrix}
    \item $\dfrac{\text{distance to foci}}{\text{distance to directrix}} = e = 1$
\end{itemize}
\subsection{Cartesian equation}
\begin{itemize}
    \item $xy=c^2$ ($c>0$)
\end{itemize}
\subsection{Parametric equation}
\begin{itemize}
    \item $x=ct$
    \item $y=\frac{c}{t}$
    \item $t \neq 0, t\in\Rset$
\end{itemize}
\subsection{Eccentricity}
\begin{itemize}
    \item $e=\sqrt{2}$
\end{itemize}
\subsection{Directrix}
\begin{itemize}
    \item $x+y=\pm c\sqrt{2}$
\end{itemize}
\subsection{Tangents and normals}
\begin{itemize}
    \item Equation of tangent: $x+t^2y=2ct$ at $P(ct, \frac{c}{t})$
    \item Equation of normal: $t^3x-ty=c(t^4-1)$ at $P(ct, \frac{c}{t})$
\end{itemize}

\chapter{Elastic strings and springs and elastic energy}
\section{Hooke's law}
\begin{itemize}
    \item $\text{Tension produced}\propto x$ $\rightarrow$ $T=kx$, where $k$ is a constant
    \item $k$ depends on the unstretched length of the string or spring ($l$) and the \textbf{modulus of elasticity of the string or spring} ($\lambda$, unit = N)
    \item Hence $T=\frac{\lambda x}{l}$
    \item[*] Can also be applied if the string or spring is compressed
\end{itemize}

\section{Elastic energy}
\begin{itemize}
    \item Work done in stretching an elastic string or spring of modulus of elasticity $\lambda$ from its natural length $l$ to a length of $(l+x)$ = $\frac{\lambda x^2}{2l}$
    \item Elastic potential energy stored = amount of energy used to stretch the spring to a length of $(l+x)$ = $\frac{1}{2}kx^2$ = $\frac{\lambda x^2}{2l}$
    \item[*] Can also be applied when an elastic string or spring is compressed
\end{itemize}


\chapter{Elastic collisions in one dimension}
\section{Vieta's Law}
For $a_{n}x^{n}+a_{n-1}x^{n-1}+\cdots +a_{1}x+a_{0}=0$:
\begin{itemize}
    \item $\sum x_i=-\frac{a_{n-1}}{a_n}$
    \item $\sum x_ix_j=\frac{a_{n-2}}{a_n}$
    \item $\sum x_ix_jx_k = -\frac{a_{n-3}}{a_n}$
    \item $\sum _{1\leq i_{1}<i_{2}<\cdots <i_{k}\leq n}\left(\prod _{j=1}^{k}r_{i_{j}}\right)=(-1)^{k}{\frac {a_{n-k}}{a_{n}}}$
    \item $\prod x_i=(-1)^n\frac{a_0}{a_n}$
\end{itemize}

\chapter{Elastic collisions in two dimensions}
\section{Oblique impact with a fixed surface}
\includegraphics[width=0.3\textwidth]{obliqueimpact}
\begin{itemize}
    \item The component of the velocity of the sphere parallel to the surface is unchanged: $v\cos\beta = u\cos\alpha$
    \item The component of the velocity of the sphere perpendicular to the surface can be found with Newton's law of restitution: $v\sin\beta = eu\sin\alpha$
    \item Hence $\tan\beta = e\tan\alpha$, since $0 \leq e \leq 1$, $\beta \leq \alpha$
    \item $\text{Loss of kinetic energy}=\dfrac{1}{2}mu^2-\dfrac{1}{2}mv^2$
\end{itemize}
\section{Oblique impact of smooth spheres}
\includegraphics[width=0.3\textwidth]{oblique_2_balls}
\begin{itemize}
    \item Impulse affecting each sphere acts along the line of centres
    \item The components of the velocities of the spheres \textbf{perpendicular}  to the line of centres are unchanged in the impact
    \item The principle of conservation of momentum and Newton's law of restitution applies \textbf{parallel} to the line of centres
\end{itemize}





\end{document}